%% Latex markup und Zitate funktionieren auch hier

{
In dieser Arbeit werden im Rahmen der Quantenelektrodynamik gebundener Zustände verschiedene Einflüsse der Kernstruktur auf die Spektren wasserstoffartiger Systeme untersucht.
Im ersten Teil dieser Arbeit wird der $g$-Faktor des gebundenen Elektrons untersucht, welcher von der Form der Ladungsverteilung im Atomkern abhängt. Ein numerischer, nicht-perturbativer Ansatz für die Berechnung der entsprechenden Kernformkorrektur zum $g$-Faktor wird vorgestellt und Implikationen für die Unsicherheiten theoretischer Vorhersagen werden diskutiert. Im Besonderen kann die Modellabhängigkeit der Kerngrößenkorrektur zum $g$-Faktor durch Nutzen von Informationen über die verformte Ladungsverteilung des Kerns verringert werden.
Im zweiten Teil geht as um die Struktur gebundener Zustände zwischen einem Myon and einem Atomkern, sogenannte myonische Atome.
Hierbei werden hochpräzise Berechnungen der Übergangsenergien und -wahrscheinlichkeiten mit zeitgemäßen numerischen Methoden durchgeführt. Des weiteren werden neue Methoden für die Berechnung von Korrekturen höherer Ordnung zu der Hyperfeinstruktur präsentiert. Dies beinhaltet eine vollständige Berechnung der Hyperfeinaufspaltung zweiter Ordnung und Korrekturen aufgrund von Vakuumpolarisationseffekten für Quadrupoleverteilungen im Innern des Atomkerns.
Es werden hochpräzise Berechnungen des $g$-Faktors des gebundenen Myons in $^{4}$He duchrgeführt und in Verbindung mit kürzlich durchgeführten Experimenten wird das Quadrupolemoment von $^{185}$Re und $^{187}$Re Kernen ermittelt.
}
