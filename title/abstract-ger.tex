%% Latex markup und Zitate funktionieren auch hier

{\small
In dieser Arbeit werden verschiedene Einflüsse der Kernstruktur und der Korrekturen der Quantenelektrodynamik (QED) auf die Spektren wasserstoffartiger Systeme untersucht.
Im ersten Teil geht as um die Struktur gebundener Zustände zwischen einem Myon and einem Atomkern, sogenannter myonischer Atome.
Hierbei werden präzise Berechnungen der Übergangsenergien und -wahrscheinlichkeiten mit zeitgemäßen numerischen Methoden durchgeführt. Die Kernausdehnung, QED Korrekturen, Hyperfeinaufspaltung and die Wechselwirkung mit Schalenelektronen werden berücksichtigt. Des weiteren werden neue Methoden für die Berechnung von Korrekturen höherer Ordnung zu der Hyperfeinstruktur präsentiert. Dies beinhaltet eine vollständige Berechnung der Hyperfeinaufspaltung zweiter Ordnung und Korrekturen aufgrund von Vakuumpolarisationseffekten für Quadrupoleverteilungen im Innern des Atomkerns. 
In Verbindung mit kürzlich durchgeführten Experimenten wird das Quadrupolemoment von $^{185}$Re und $^{187}$Re Kernen ermittelt.
Im zweiten Teil dieser Arbeit wird der $g$-Faktor des gebundenen Elektrons untersucht, welcher von der Form der Ladungsverteilung im Atomkern abhängt. Ein numerischer, nicht-perturbativer Ansatz für die Berechnung der entsprechenden Kernformkorrektur zum $g$-Faktor wird vorgestellt und Implikationen für die Unsicherheiten theoretischer Vorhersagen werden diskutiert. Im Besonderen kann die Modellabhängigkeit der Kerngrößenkorrektur zum $g$-Faktor aufgrund des besseren Modells für die Ladungsverteilung im Kern verringert werden.
Des weiteren tragen Berechnungen der Kerngrößen- und Vakuumpolarisationskorrekturen für den $g$-Faktor des gebundenen Myons in $^{4}_2$He zu einer Vorhersage auf dem $10^{-9}$~Niveau bei. Ein experimenteller Wert mit derselben Genauigkeit könnte eine genauere Ermittlung der Myonmasse oder des magnetischen Moments des Myon ermöglichen.
}
