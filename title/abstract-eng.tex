%% Latex markup and citations may be used here

{\small
In this thesis, several aspects of nuclear structure effects and corrections from quantum electrodynamics (QED) in the spectra of hydrogen-like systems are investigated. The first part of this thesis is about the $g$ factor of a bound electron and its dependence on the shape of the nuclear charge distribution. 
A numerical, non-perturbative approach for the calculation of the corresponding nuclear shape correction is presented and implications for the uncertainties of theoretical predictions are discussed. In particular, the model-uncertainty of the finite-nuclear-size correction to the $g$ factor can be reduced using information on the deformed charge distribution.
The second part is concerned with the structure of bound states between a muon and an atomic nucleus, so called muonic atoms. Here, precise calculations for transition energies and probabilities are presented, using up-to-date numerical methods. The finite nuclear size effect, QED corrections, hyperfine interactions, and the interaction with atomic electrons were evaluated. 
Furthermore, new methods for the calculation of higher order corrections for the hyperfine structure are presented, including a complete calculation of the second order hyperfine structure and leading order vacuum polarization corrections for extended electric quadrupole distributions inside the nucleus. 
In connection with recent measurements on muonic atoms, the nuclear quadrupole moment of $^{185}$Re and $^{187}$Re is extracted.
Furthermore, calculations of finite-size and vacuum-polarization corrections to the $g$ factor of a muon bound to a $^{4}_2$He nucleus contribute to a theoretical prediction on the $10^{-9}$ level. An experimental value of the same accuracy could give access to an improved value of the muon's mass or magnetic moment anomaly.
}

\thispagestyle{empty}
