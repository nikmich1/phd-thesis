\chapter*{Introduction}
\markboth{Introduction}{}
Advances in the field of spectroscopy have always given new insights in the physical laws which govern our world at the smallest scales. %or the microcosm
The first observation of a discrete absorption spectrum was due to Wollaston in 1802~\cite{wollaston1802} and Fraunhofer in 1814~\cite{fraunhofer1817}, who discovered the Fraunhofer lines in the spectrum of the sun independently from each other.
In the following decades, the emission spectra of different elements were explored, especially noteworthy are the systematic investigations by Kirchhoff and Bunsen~\cite{kirchhoff1860,kirchhoff1861}. It became apparent that elements can be identified by their characteristic spectrum and that laboratory emission spectra are connected to astrophysical absorption spectra~\cite{angstrom1862}.
Since the hydrogen atom consists of only one electron bound to a single proton, it has the simplest spectrum and therefore was particularly important for the development of theoretical models. It was recognized by Balmer in 1885~\cite{balmer1885} that the position of spectral lines as measured by Ångström~\cite{angstrom1853} and Vogel~\cite{vogel1880} could be described with surprising accuracy by a simple formula. This was generalized later in terms of the Rydberg formula~\cite{rydberg1889,martinson2005}. It contains the Balmer series as a special case and also predicts the Lyman, Paschen, Brackett, Pfund, and Humphreys series, which were confirmed subsequently by experiments~\cite{lyman1906,paschen1908,brackett1922,pfund1924,humphrey1953}.
















~\cite{haensch1979}
















\clearpage
\subsubsection*{Spectroscopy of atoms, picture of atoms in general}
\begin{itemize}
\item
review picture of atom and structure of atom (beginning with rutherford's experiment)\\
review history of spectroscopy of atoms and connection to development of theories\\
(discrete spectrum -> quantum mechanics; finestructure -> dirac eq.; lambshift,hfs -> finite nuclear size, qed). \\
Also mention recent development (H precision spectroscopy, highly charged ions, x ray spectr., microwave spectr.)\\
\end{itemize}

\subsubsection*{contemporary research}
\begin{itemize}
\item recent (precision) spectroscopy
\item especially heavy ions, nuclear structure effects in spectra
\end{itemize}

\subsubsection*{bound electron g factor}
\begin{itemize}
\item
history of measurements and theory (Zeeman). recent project (heaviest highly charged ions). penning traps, fine structure constant, electron mass, specific difference etc.
\end{itemize}

\subsubsection*{muonic atoms}
\begin{itemize}
\item
explain field of exotic atoms (positronium, muonium, true muonium, pionic atoms, muonic atoms) and which regimes can be tested with them. Give more details about muonic atoms, history of findings and results. Describe the new experiments at psi.
\item 
explain proton radius puzzle as a motivation for more experiments on muonic systems, also g-2 anomlies for muon
\end{itemize}



\subsubsection*{Structure of the Thesis}

Explain the structure here.




