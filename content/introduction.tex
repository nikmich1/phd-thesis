\chapter*{Introduction}
\markboth{Introduction}{}
Advances in spectroscopy have always given new insights into the physical laws which govern our world at the smallest scales.
The first observation of a discrete absorption spectrum was due to Wollaston in 1802~\cite{wollaston1802} and Fraunhofer in 1814~\cite{fraunhofer1817}, who discovered the Fraunhofer lines in the solar spectrum independently from each other.
In the following decades, the emission spectra of different elements were explored. Especially noteworthy are the systematic investigations by Kirchhoff and Bunsen in Heidelberg~\cite{kirchhoff1860,kirchhoff1861}. It became apparent that elements can be identified by their characteristic spectrum and that laboratory emission spectra are connected to astrophysical absorption spectra~\cite{angstrom1862}.

Since the hydrogen atom consists of only one electron bound to a single proton, it has the simplest spectrum among all atoms and therefore was particularly important for the development of theoretical models. It was recognized by Balmer in 1885~\cite{balmer1885} that the position of spectral lines as measured by Ångström~\cite{angstrom1853}, Huggins~\cite{huggins1880}, and Vogel~\cite{vogel1880} could be described with surprising accuracy by a simple formula. This was generalized later in terms of the Rydberg formula~\cite{rydberg1889,martinson2005}. It describes the Balmer series as a special case and also predicts the Lyman, Paschen, Brackett, Pfund, and Humphreys series, which were confirmed subsequently by experiments~\cite{lyman1906,paschen1908,brackett1922,pfund1924,humphrey1953}. However, the Rydberg formula is purely empirical, without an underlying theoretical framework.

Additionally, the electron was discovered by the investigation of cathode rays~\cite{thomson1897,rechenberg1997}, and Rutherford scattering showed that the positive charge and almost the entire mass of an atom is concentrated in its center in form of an atomic nucleus~\cite{rutherford1911}. Also, the spectral density of black-body radiation was explained by Planck using the quantum hypothesis~\cite{planck1978}. This motivated the Bohr model of the atom~\cite{bohr1913}, according to which the electron can revolve around the nucleus only on certain quantized orbits. Compared to the previous Thomson~\cite{thomson1904} and Rutherford model, now the Rydberg formula and thereby the hydrogen spectrum could be derived, and the discrete energies could be expressed in terms of the fine-structure constant~$\alpha$, the electron mass~$m_e$, and the speed of light~$c_0$. The relativistic extension of this model is the Bohr-Sommerfeld model~\cite{sommerfeld1916}, which explains also finer features of the hydrogen spectrum.
However, despite the success of describing the quantized energies, the Bohr-Sommerfeld model has difficulties with the generalization to many-electron systems. A consistent theoretical framework for non-relativistic atomic theory, also for more complicated atoms, was finally obtained with the Schrödinger equation~\cite{schrodinger1926_1,schrodinger1926_2,schrodinger1926_3,schrodinger1926_4} and matrix mechanics~\cite{heisenberg1925,born1925,born1926}, which were shown to be equivalent formulations of quantum theory~\cite{schrodinger1926_5}.

Due to the Zeeman effect~\cite{zeeman1896}, spectral lines of atoms exposed to an external magnetic field are split into sublevels. This could only be explained consistently by assigning, besides the orbital angular momentum, also the spin angular momentum to the electron~\cite{uhlenbeck1925}. The Dirac equation~\cite{dirac1928} incorporates the electron's spin naturally and predicts that the corresponding magnetic moment due to spin is twice as large as the orbital angular momentum. Additionally, Dirac's equation obeys the laws of special relativity, and among its solutions negative energy states occur, which led to the prediction of the positron, the electron's antiparticle. The negative-energy states also lead to problems with the one-particle interpretation of the Dirac equation due to phenomena such as the instability of the hydrogen ground state, Zitterbewegung, and Klein's paradox~\cite{the_dirac_eq}.
 
Two experimental results pointed out that the Dirac equation, despite its success in describing the energy levels of the hydrogen atom, could not be the end of the story for the theory of atomic structure. 
The Dirac equation for a point-like nucleus predicts that two energy levels with the same principal quantum number and total angular momentum are degenerate~\cite{greiner2000}. Therefore, the $2s_{1/2}$ and $2p_{1/2}$ levels should be degenerate according to Dirac's theory. However, Lamb and Retherford showed for hydrogen that these levels are separated by about $1060\,\rm{MHz}$ by driving the transition directly with radio waves~\cite{lamb1947}. On the other hand, anomalies in the magnetic hyperfine structure of hydrogen and deuterium~\cite{nafe1947} as well as sodium and gallium~\cite{kusch1947,kusch1948} were revealed. Both phenomena were explained in the framework of quantum electrodynamics (QED)~\cite{schwinger1948}, which yields small corrections to the energy levels in atoms and to the magnetic moment of the electron. To this date, the comparison of experiment and theory for the hyperfine and Zeeman splitting in simple atomic systems keeps challenging QED and delivering values for fundamental physical constants~\cite{haensch1979}. The following two sections describe more recent developments in this field.

\subsubsection*{Muonic atoms}
For further studies of H-like systems, either the atomic nucleus or the bound electron can be exchanged with another charged particle. This establishes the field of research on \textit{exotic atoms}. Depending on the type of considered particles, nuclear structure effects can either be avoided or enhanced. Bound states between two leptons are not affected by the strong interaction or nuclear effects and are therefore suitable for tests of bound-state QED in a cleaner environment. 
% positronium
One example is positronium~\cite{karshenboim2004}, a system formed by an electron and its antiparticle, the positron. The energy levels were studied up to order $\alpha^6$ in Refs.~\cite{pineda1998,pachucki1998,czarnecki1999,zatorski2008} and measured on a $10^{-4}$ up to $10^{-9}$ level, e.g. in Refs.~\cite{mills1975,ritter1984,danzmann1989,hagena1993,fee1993}, despite a lifetime in the range of $10^{-9}$s due to pair annihilation. 
% muonium
Another interesting leptonic system is \textit{muonium}~\cite{jungmann2004}, a bound state consisting of an antimuon and an electron. The antimuon is the positively charged lepton in the second generation of matter in the Standard Model of particle physics and the muon the negatively charged one. Since the mass ratio of muon and proton is $m_p/m_\mu \approx 8.9$, the spectrum of muonium is quite similar to that of hydrogen, except that effects due to proton structure are avoided. The hyperfine splitting in muonium was measured to a part-per-billion level in Refs.~\cite{casperson1975,liu1999} and calculated with a similar accuracy in Refs.~\cite{pachucki1996,Karshenboim1996,sapirstein1997,nio1997,eides1998}. The \textit{MuSEUM} collaboration~\cite{museum} plans new precision experiments on the hyperfine structure in muonium. 
% true muonium / dimuonium
The muon-antimuon bound state is called \textit{true muonium}~\cite{brodsky2009}, or \textit{dimuonium}. It has yet to be observed, which is the aim of the $\mu\mu$-tron collider, which is constructed at the Budker Institute of Nuclear Physics (Russia)~\cite{bogomyagkov2017}. 
% true tauonium
A similar bound state exists in principle also for third-generation of matter, a tau-antitau bound state called \textit{true tauonium}. However, due to the extremely short lifetime of tau leptons, it would be even more difficult to observe~\cite{brodsky2009}.


%muonic atoms
The negatively charged muon can also form bound states with atomic nuclei. These exotic atoms are commonly referred to as \textit{muonic atoms}. In this case, the nuclear structure effects are greatly enhanced for the following reason: The muon is about 207 times heavier than the electron. Therefore, the Bohr radius of the muonic orbitals, which is an estimate for the distance between bound particle and nucleus, is also 207 times smaller. As a consequence, especially for high $Z$, the bound-muon wave function has a large overlap with the nuclear charge distribution, which results in remarkable interplays between atomic and nuclear physics. Observation of X-rays from bound-bound transitions in muonic atoms was reported for the first time in Ref.~\cite{chang1949} in cosmic ray studies, and in Ref.~\cite{fitch1953} with a laboratory muon beam. The beginning of muonic atom theory is marked by the seminal paper by Wheeler~\cite{wheeler1949}.  Since then, the spectra of muonic atoms have been investigated in numerous experiments, e.g.~\cite{hitlin1970,zehnder1975,powers1976,Yamazaki1978,tanaka1983,tanaka1984,tanaka1984_2,Bergem1988,
powers1977}. In particular, absolute RMS charge radii of atomic nuclei were obtained by analysis of muonic x-ray spectra for the majority of stable nuclei~\cite{FRICKE1995}. An overview over the progress in theoretical calculations can be found in Refs.~\cite{BorieRinker1982,Devons1995,wu1969}. In the past, the codes \texttt{MUON} and \texttt{RURP}~\cite{rinker1979} based on simple solutions of the Dirac equations and developed in the 1970s have been used frequently for the comparison of theoretical predictions and measurements of muonic x-rays. 

Muonic hydrogen came to attention recently because laser spectroscopy of a \mbox{\small{$2p\rightarrow 2s$}} transition enabled the extraction of the proton charge radius as $r_p^{(\mu)}=0.84184(67)\,\rm{fm}$, and the result turned out to be smaller than the CODATA value $0.8768(69)\,\rm{fm}$ at the time by $5.0$ standard deviations~\cite{Pohl2010}. In 2013, a new measurement increased the deviation to 7 standard deviations~\cite{antognini2013}. Measurements on muonic deuterium confirmed these results~\cite{pohl2016} for the deuterium charge radius and lately, also an experiment with atomic hydrogen measured the small proton radius~\cite{beyer2017}.
Results from elastic electron-proton scattering, which are also used for the CODATA value~\cite{codata}, seem to confirm the larger value for the proton radius, but the proton-radius extraction from scattering data is not unambiguous~\cite{arrington2015}. However, a new result from atomic hydrogen spectroscopy resulted in the larger value again~\cite{fleurbaey2018}, thus this \textit{proton radius puzzle} is not resolved until now. Together with anomalies on the magnetic moment of the muon~\cite{bennett2006}, this motivates further investigation of muonic systems.

In the high-$Z$ regime, the MuX collaboration at the Paul Scherrer Institut (Switzerland) has recently started to revive x-ray spectroscopy of muonic atoms after these kind of measurements have not been performed for nearly thirty years.
The aim is to measure muonic x-ray spectra for the heaviest and also radioactive nuclei, for example $^{226}_{\phantom{1}88}$Ra and $^{248}_{\phantom{1}96}$Cm, and to extract information on nuclear parameters. 

\subsubsection*{The $g$ factor of the bound electron}
The magnetic moment of the electron is commonly expressed by the dimensionless gyromagnetic factor, or $g$ factor, which is the proportionality constant between magnetic moment and angular momentum. Experiments on the $g$ factor of the free electron provide one of the most stringent tests of QED without a strong external electromagnetic field. It was measured with an uncertainty below the part-per-trillion level~\cite{odom2006,hanneke2008} and predicted to order $\alpha^5$ theoretically, e.g.~  \cite{schwinger1948,Peterman57,Sommerfield1957,Sommerfield58,Laporta96,kinoshita2006,aoyama2007,aoyama2015,aoyama2017}. A combination of experiment and theory has allowed the extraction of the fine-structure constant $\alpha$ on the parts-per-billion level~\cite{gabrielse2006,gabrielse2007}.

The $g$ factor can also be measured and theoretically calculated to an extraordinary precision in case of the electron bound in a highly charged ion. Here, QED can be tested in the regime of strong fields, since the electron is exposed to the nuclear Coulomb potential. It has been measured for H-like $^{12}_{\phantom{0}6}\rm{C}^{5+}$~\cite{Haffner2000,Sturm2014}, $^{16}_{\phantom{0}8}\rm{O}^{7+}$~\cite{Verdu2004}, and $^{28}_{14}\rm{Si}^{13+}$~\cite{Sturm2011}, and for Li-like $^{28}$Si$^{11+}$~\cite{sturm2013}, $^{40}$Ca$^{17+}$~\cite{Kohler2016}, and $^{48}$Ca$^{17+}$~\cite{Kohler2016} on the part-per-billion level with Penning trap experiments performed in Mainz (Germany) using a single trapped ion. A Penning trap is a device for trapping charged particles with a combination of a static electric quadrupole field and a homogeneous magnetic field~\cite{annphysgfactor,geoniumtheory}. 

Two developing experiments aim at measuring the bound-electron $g$ factor in H-like ions also for very high charge numbers. The \textit{ALPHATRAP} experiment~\cite{sturm2017} uses a measurement scheme with two Penning traps similar to the \textit{Mainz $g$-Factor Experiment}. Now, the ions are not created \textit{in situ} but can be injected from external sources, like the \textit{Heidelberg EBIT} (electron-beam ion trap)~\cite{ebit1999}. In this way, $g$ factors of ions up to H-like $^{208}\rm{Pb}^{81+}$ can be investigated with an expected accuracy of 10 parts-per-trillion~\cite{sturm2017}. The \textit{ARTEMIS} experiment at the GSI Darmstadt (Germany)~\cite{vogel2013,sturm2017} will investigate the structure of Zeeman sublevels in highly charged ions. In this way, ground- and excited-state $g$ factors can be accessed. With connection to the \textit{HITRAP} beamline, the heaviest hydrogen-like ions, e.g. $^{238}\rm{U}^{91+}$, will be available~\cite{vogel2015}.

The precision experiments on the bound-electron $g$ factor demand theoretical calculations on a competing level of accuracy. The interaction of a bound electron with the atomic nucleus is characterized by the parameter $Z\alpha$, where $Z$ is the nuclear charge number and $\alpha \approx 1/137$. For light nuclei, $Z\alpha$ is a small parameter and corrections to the binding energies can be calculated by a perturbative expansion in this parameters. On the other hand, for heavy nuclei, $Z\alpha$ is on the order of unity, and a power series expansion of energy corrections in $Z\alpha$ is not always viable. %
%Breit term
The leading contribution to the binding corrections to the bound-electron $g$ factor is due to the point-like Coulomb potential and has been obtained by Breit~\cite{breit1928}.
%QED
The one- and two-loop vacuum-polarization (VP) and self-energy (SE) corrections have been calculated to order $(Z\alpha)^4$ in Refs.~\cite{karshenboim2000,Pachucki2004,pachucki2004_err,Pachucki2005,czarnecki2016}. 
Two-loop corrections to order $(Z\alpha)^5$ have been recently presented in Ref.~\cite{czarnecki2018}.
One-loop QED corrections for the VP and SE to all orders in $Z\alpha$ has been calculated in Refs.~\cite{Beier2000,Karshenboim2001,yerokhin2002,Yerokhin2004,Lee2005,Lee2007,yerokhin2008,yerokhin2010,yerokhin2017}.
Two-loop calculations to all orders in $Z\alpha$ have not been completed to date. They have been presented for two VP loops and for the mixed VP-SE effect in Ref.~\cite{yerokhin2013}, and for the SE loop-after-loop terms in Ref.~\cite{sikora2018_arxiv}.

%nuclear effects
Furthermore, nuclear effects beyond those due to the point-like Coulomb potential have to be considered. Although the nucleus is much smaller than the typical extent of the electron wave function, it is an extended object and, correspondingly, the Coulomb potential is modified at small distances. This causes the finite nuclear size correction to the energy levels and to the $g$ factor of the bound electron. A relativistic analytic formula for this effect has been given in Ref.~\cite{Glazov2002} and the corresponding non-relativistic limit in Ref.~\cite{karshenboim2000}. In case of two-photon exchanges between the bound electron and internal nuclear currents, also excited nuclear states contribute, leading to the nuclear polarization correction. This has been considered in Refs.~\cite{Nefiodov,volotka2014}. 
%recoil
The calculation of all effects mentioned so far assumed an infinitely heavy resting nucleus. The nuclear recoil corrections account for the finite nuclear mass. Here, besides $\alpha$ for QED loops and $Z\alpha$ for interactions with the nuclear potential, an additional expansion parameter $m_e/M$ appears, which is the electron-to-nucleus mass ratio. To order $\alpha(Z\alpha)^2(m_e/M)^2$, results can be found in Refs.~\cite{eides2010} and to all orders in $m_e/M$ and first order in $Z\alpha$ in Ref.~\cite{pachucki2010}. Suitable for heavy ions, the recoil correction to first order in $m_e/M$, but to all orders in $Z\alpha$, is given in Refs.~\cite{shabaev2001,Shabaev2002}.
%introduce nuclear shape and motivate work in this thesis
In Ref.~\cite{jacek2012}, the nuclear shape effect, also called nuclear deformation effect, was introduced for spinless nuclei, which takes the deformation, i.e. the deviation from a perfect spherically symmetric shape of the nuclear charge distribution into account. This contribution is not significant for light nuclei at the current level of experimental accuracy. However, it scales strongly with the nuclear charge and therefore becomes important for high $Z$. 

%outlook/outline
The combination of theory and experiment for the bound-electron $g$ factor in $^{12}_{\phantom{0}6}$C provided an improved value of the electron mass~\cite{Kohler2015,Sturm2014,Zatorski2017}. For $^{28}_{14}$Si~\cite{Sturm2011}, it was shown as a proof-of-principle determination that nuclear parameters like the RMS charge radius can be obtained. Also, the extraction of nuclear magnetic moments was suggested theoretically~\cite{Yerokhin2011,Werth2001}. Furthermore, it was argued that an independent and more accurate value for the fine-structure constant can be obtained~\cite{Shabaev2006,yerokhin2016,Yerokhin2016PRA}. With upcoming experiments in the high-$Z$ regime, further tests of QED in strong fields, new information on nuclear parameters, and the extraction of fundamental constants can be expected, and improved theoretical calculations especially for heavy nuclei are needed.

\subsubsection*{The subject of this thesis}

%muonic atoms
This thesis presents calculations of spectra of muonic atoms with new methods for predicting higher order effects and up-to-date numerical schemes. 
The calculation of transition energies and transition probabilities in heavy muonic atoms is necessary for the comparison with the above-mentioned recent experiments by the MuX collaboration and the extraction of nuclear parameters.

In muonic atoms, the connection between measured spectra and nuclear parameters is obscured due to the highly complicated level structure. Complex computations need to be performed for the generation of the spectrum for a given nuclear charge distribution. Conversely, the theoretically calculated spectrum can be matched to the experimentally measured one by adjustment of the nuclear parameters in the computations. In this way, nuclear parameters can be extracted. Correspondingly, for the fitting process, all theoretical calculations need to be unified in one single approach. 
During the last years, the dual-kinetic-balance (DKB) method~\cite{Shabaev2004} has proved to be a very successful numerical approach in relativistic atomic structure calculations, but has not been used to date for muonic atoms.
In this thesis, the DKB method based on B-spline representations of wave functions is used to calculate the energy levels of the bound muon in the electric field of heavy nuclei, where contributions due to the finite nuclear size, QED corrections, hyperfine interactions, and electron screening are readily included.
With the DKB method, a direct numerical evaluation of second-order energy corrections is possible by a summation over the complete spectrum of the bound muon. This is demonstrated in the thesis with the second-order electric quadrupole interaction and it is shown that this contribution is important in experiments with very heavy nuclei.

The vacuum polarization (VP) correction due to a virtual electron-positron pair gives a sizable correction to energy levels in muonic atoms. This correction also affects electric multipole operators, in particular, the electric quadrupole interaction. In this thesis, it is shown how matrix elements of multipole potentials of any order due to an arbitrary deformed nuclear charge distribution can be systematically corrected for the leading-order~VP. For the quadrupole interaction, analytical expressions in terms of special functions have been derived.  Numerical studies for uranium and rhenium are presented in this thesis.

The calculations are compared to measurements of spectra of isotopically pure muonic rhenium, which were recorded in 2016 at the Paul Scherrer Institute (Switzerland). A comparison of theory and experiment enabled the extraction of the nuclear quadrupole moment. In this thesis, a fitting procedure with the consideration of effects non-linear in the quadrupole moment is constructed.
With planned measurements on elements as heavy as $^{248}\text{Cm}$ in the near future, extraction of further nuclear parameters with this approach can be expected.\\[11pt]%
%electron g factor
In addition, nuclear effects on the bound electron $g$ factor were considered in this thesis. To date, the nuclear deformation correction was calculated with perturbative methods only. This thesis contributes to the theory landscape by investigating the nuclear deformation correction non-perturbatively with advanced numerical methods. It is shown that the perturbative approach overestimated the nuclear deformation effect by about 20\% and the reason for the difference is analyzed.\\[11pt]%
%muon g factor
Furthermore, in this thesis the finite nuclear size effect and several one- and two-loop QED corrections for the bound-muon $g$ factor in helium are calculated, namely, the all-order Uehling and the Källén-Sabry terms. Together with calculations of further contributions by other authors, this enabled a theoretical prediction of the $g$ factor on a $10^{-9}$ level. This can potentially give access to an improved determination of the muon's mass or magnetic moment anomaly, provided that a measurement of similar accuracy could be performed.


\subsubsection*{The structure of the thesis}

This thesis is organized in the following way:\\
Chapter~\ref{ch:furry_pic} gives an introduction to the framework of bound-state QED. 
The Dirac equation is discussed, and analytical and numerical solutions for nuclear potentials and vacuum polarization potentials are given.\\
In Chapter~\ref{ch:muonic_atoms}, the energy levels and transition probabilities of muonic atoms are obtained. An improved numerical spectrum generator for the extraction of nuclear parameters is presented in Section~\ref{sec:calculationSpectraMuon}, as well as new methods and calculations for vacuum polarization and higher-order hyperfine splitting effects in Section~\ref{sec:higherorder}. \\
Then, in connection with measurements on isotopically pure muonic rhenium, the nuclear spectroscopic quadrupole moment is extracted in Section~\ref{sec:muon_re}.\\
In Chapter~\ref{ch:nucl_def}, the nuclear finite size and deformation corrections to the bound-electron $g$ factor are examined. A numerical method for their precise calculation is introduced and results are compared to previous studies.\\
For muonic helium, the high-precision calculations of the finite nuclear-size and one- and two-loop vacuum-polarization corrections are described in Chapter~\ref{sec:muon_he}.\\
Finally, the main findings of the thesis are summarized and an outlook is given.




































%\clearpage
%
%
%
%\begin{itemize}
%\item a bit more history for free and bound el. g factor experiment (eg. v. Dyck., Dehmelt)
%\item cite w.q.'s papers
%\item problems with 1-particle interpretation of the Dirac eq. resolved due to QED point of view
%\item connection to nuclear deformation and this thesis
%\item muonic atoms: one possibility for measuring absolute charge radii
%\item for outlook: generalization of nuclear shape correction to non-zero spins
%\end{itemize}
%[heavy ions]\\[11pt]
%[exotic atoms]\\[11pt]
%
%\subsubsection*{Spectroscopy of atoms, picture of atoms in general}
%\begin{itemize}
%\item
%review picture of atom and structure of atom (beginning with rutherford's experiment)\\
%review history of spectroscopy of atoms and connection to development of theories\\
%(discrete spectrum -> quantum mechanics; finestructure -> dirac eq.; lambshift,hfs -> finite nuclear size, qed). \\
%Also mention recent development (H precision spectroscopy, highly charged ions, x ray spectr., microwave spectr.)\\
%\end{itemize}
%
%\subsubsection*{contemporary research}
%\begin{itemize}
%\item recent (precision) spectroscopy
%\item especially heavy ions, nuclear structure effects in spectra
%\end{itemize}
%
%\subsubsection*{bound electron g factor}
%\begin{itemize}
%\item
%history of measurements and theory (Zeeman). recent project (heaviest highly charged ions). penning traps, fine structure constant, electron mass, specific difference etc.
%\end{itemize}
%
%\subsubsection*{muonic atoms}
%\begin{itemize}
%\item
%explain field of exotic atoms (positronium, muonium, true muonium, pionic atoms, muonic atoms) and which regimes can be tested with them. Give more details about muonic atoms, history of findings and results. Describe the new experiments at psi.
%\item 
%explain proton radius puzzle as a motivation for more experiments on muonic systems, also g-2 anomlies for muon
%\end{itemize}





