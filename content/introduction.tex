\chapter*{Introduction}
\markboth{Introduction}{}
Advances in the field of spectroscopy have always given new insights in the physical laws which govern our world at the smallest scales. %or the microcosm
The first observation of a discrete absorption spectrum was due to Wollaston in 1802~\cite{wollaston1802} and Fraunhofer in 1814~\cite{fraunhofer1817}, who discovered the Fraunhofer lines in the spectrum of the sun independently from each other.
In the following decades, the emission spectra of different elements were explored. Especially noteworthy are the systematic investigations by Kirchhoff and Bunsen~\cite{kirchhoff1860,kirchhoff1861}. It became apparent that elements can be identified by their characteristic spectrum and that laboratory emission spectra are connected to astrophysical absorption spectra~\cite{angstrom1862}.
Since the hydrogen atom consists of only one electron bound to a single proton, it has the simplest spectrum and therefore was particularly important for the development of theoretical models. It was recognized by Balmer in 1885~\cite{balmer1885} that the position of spectral lines as measured by Ångström~\cite{angstrom1853}, Huggins~\cite{huggins1880}, and Vogel~\cite{vogel1880} could be described with surprising accuracy by a simple formula. This was generalized later in terms of the Rydberg formula~\cite{rydberg1889,martinson2005}. It contains the Balmer series as a special case and also predicts the Lyman, Paschen, Brackett, Pfund, and Humphreys series, which were confirmed subsequently by experiments~\cite{lyman1906,paschen1908,brackett1922,pfund1924,humphrey1953}. However, the Rydberg formula is purely empirical, without an underlying theoretical framework.

Additionally, the electron was discovered by the investigation of cathode rays~\cite{thomson1897,rechenberg1997} and Rutherford scattering showed that the positive charge and almost the entire mass of an atom is concentrated in its center in form of an atomic nucleus~\cite{rutherford1911}. Also, the spectral density of black-body radiation was explained by Planck using the quantum hypothesis~\cite{planck1978}. This motivated the Bohr model of the atom~\cite{bohr1913}, where the electron can orbit the nucleus only on certain quantized orbits. Compared to the previous Thomson~\cite{thomson1904} and Rutherford model, now the Rydberg formula and thereby the hydrogen spectrum could be derived and the discrete energies could be expressed in terms of the fine-structure constant $\alpha$, the electron mass $m_e$, and the speed of light $c$. A relativistic extension is the Bohr-Sommerfeld model~\cite{sommerfeld1916}, which explains also finer features of the hydrogen spectrum.
However, despite the success of describing the quantized energies, the Bohr-Sommerfeld model has difficulties with the generalization to many-electron systems. A consistent theoretical framework for non-relativistic atomic theory, also for more complicated atoms, was finally obtained with the Schrödinger equation~\cite{schrodinger1926_1,schrodinger1926_2,schrodinger1926_3,schrodinger1926_4} and matrix mechanics~\cite{heisenberg1925,born1925,born1926}, which were shown to be an equivalent formulation of quantum theory~\cite{schrodinger1926_5}.
Due to the Zeeman effect~\cite{zeeman1896}, spectral lines of atoms exposed to magnetic fields are split into sublevels. This could only be explained consistently by assigning, besides the orbital angular momentum, also the spin angular momentum to the electron~\cite{uhlenbeck1925}. The Dirac equation~\cite{dirac1928} incorporates the electron's spin naturally and predicts that the corresponding magnetic moment due to spin is twice as big compared to orbital angular momentum. Additionally, it obeys the laws of special relativity and negative energy states occur, which led to the prediction of the positron, the electron's antiparticle. The negative energy states also lead to problems with the one-particle interpretation of the Dirac equation due to phenomena like Zitterbewegung or Klein's paradox~\cite{the_dirac_eq}.
 
Two experimental results pointed out that the Dirac equation, despite its success in describing the energy levels of the hydrogen atom, could not be the end of the story for the theory of atomic structure. 
The Dirac equation for a point-like nucleus predicts that two energy levels with the same total angular momentum are degenerate~\cite{greiner2000}. Therefore, the $2s_{1/2}$ and $2p_{1/2}$ levels should be degenerate according to Dirac's theory. However, Lamb and Retherford showed that they are separated by about $1060\,\rm{MHz}$ by driving the transition directly with radio waves~\cite{lamb1947}. On the other hand, anomalies in the magnetic hyperfine structure of hydrogen and deuterium~\cite{nafe1947} as well as sodium and gallium~\cite{kusch1947,kusch1948} were revealed. Both phenomena were explained in the framework of quantum electrodynamics (QED)~\cite{schwinger1948}, which results in small corrections to the energy levels in atoms and the magnetic moment of the electron. To this date, the comparison of experiment and theory for the hyperfine and Zeeman splitting in simple atomic systems keeps challenging QED and delivering values for fundamental physical constants~\cite{haensch1979}. The following two paragraphs describe more recent developments in this field.
\subsubsection*{$g$ factor of the bound electron}
The magnetic moment of the electron is commonly expressed by the dimensionless gyromagnetic factor, or $g$ factor, which is the proportionality constant between magnetic moment and angular momentum. The $g$ factor of the free electron provides a stringent test of QED without an external electromagnetic field. It was measured with an uncertainty below the part-per-trillion level~\cite{odom2006,hanneke2008} and predicted to order $\alpha^5$ theoretically, eg.~\cite{kinoshita2006,aoyama2007,aoyama2015,aoyama2017}. A combination of experiment and theory allowed the extraction of the fine-structure constant $\alpha$ on the parts-per-billion level~\cite{gabrielse2006,gabrielse2007}.

The $g$ factor of the electron can also be measured and theoretically calculated to an extraordinary precision in case of the bound electron in a hydrogen-like (H-like) ion. Here, QED can be tested in the regime of high background fields, since the electron is exposed to the nuclear Coulomb potential. It has been measured for H-like $^{12}_{\phantom{0}6}\rm{C}^{5+}$~\cite{Haffner2000,Sturm2014}, $^{16}_{\phantom{0}8}\rm{O}^{7+}$~\cite{Verdu2004}, and $^{28}_{14}\rm{Si}^{13+}$~\cite{Sturm2011} on the part-per-billion level with Penning trap experiments using a single ion during the \textit{Mainz $g$-Factor Experiment}. A Penning trap is a device for trapping charged particles with a combination of a static electric quadrupole field and a homogeneous magnetic field~\cite{geoniumtheory}. For these measurements, at least two Penning traps are used~\cite{annphysgfactor}.

Two experiments aim at measuring the bound electron $g$ factor in H-like ions also for very high charge numbers. The \textit{ALPHATRAP} experiment~\cite{sturm2017} uses a measurement scheme with two Penning traps similar to the \textit{Mainz $g$-Factor Experiment}. Now, the ions are not created in situ but can be injected from external sources, like the \textit{Heidelberg EBIT} (electron-beam ion trap). In this way, $g$ factors of ions up to H-like $^{208}\rm{Pb}^{81+}$ can be investigated with an expected accuracy of 10 parts-per-trillion~\cite{sturm2017}. The \textit{ARTEMIS} experiment~\cite{vogel2013,sturm2017} will use microwaves to excite transitions between Zeeman sublevels in highly charged ions. In this way, ionic and bound electron $g$ factors can be accessed. With connection to the \textit{HITRAP} beamline, the heaviest hydrogen-like ions, e.g. $\rm{U}^{91+}$, will be available~\cite{vogel2015}.

The precision experiments on the bound electron $g$ factor demand theoretical calculations on a competing level of accuracy. The interaction of a bound electron with the atomic nucleus is characterized by the parameter $Z\alpha$, where $Z$ is the nuclear charge number and $\alpha \approx 1/137$ is the fine-structure constant. For light nuclei, $Z\alpha$ is a small parameters and a power-series expansion is feasible. On the other hand, for heavy nuclei $Z\alpha$ is on the order of unity and results have to be obtained including the nuclear Coulomb potential to all orders. %
%Breit term
The leading contribution to the binding corrections of the bound-electron $g$ factor is due to the point-like Coulomb potential and has been obtained by Breit~\cite{breit1928}.
%QED
The one- and two-loop vacuum-polarization (VP) and self-energy (SE) corrections have been obtained to order $(Z\alpha)^4$ in Refs.~\cite{karshenboim2000,Pachucki2004,pachucki2004_err,Pachucki2005}. 
Two-loop corrections to order $(Z\alpha)^5$ have been recently presented in Ref.~\cite{czarnecki2018}.
The most precise one-loop QED correction for the SE to all orders in $Z\alpha$ has been calculated in Ref.~\cite{yerokhin2017}.
Two-loop calculations to all orders in $Z\alpha$ have not been completed to date. The have been presented for two VP loops and mixed VP-SE in Ref.~\cite{yerokhin2013}, and for the SE loop-after-loop terms in Ref.~\cite{sikora2018_arxiv}.

%nuclear effects
Furthermore, nuclear effects beyond the point-like Coulomb potential have to be considered. Although the nucleus is much smaller than the typical extend of the electron wave function, it is an extended object and correspondingly, the Coulomb potential is modified at small distanced. This causes the finite nuclear size correction to the $g$ factor of the bound electron. The relativistic analytic formula for this has been given in Ref.~\cite{Glazov2002} and the corresponding non-relativistic limit in Ref.~\cite{karshenboim2000}. In two-photon exchanges, also excited nuclear states contribute and the nuclear polarization corrections have been considered in Refs.~\cite{Nefiodov,volotka2014}. 
%recoil
So far, all effects calculations assumed an infinitely heavy nuclei, which is not moving. The nuclear recoil corrections account for the finite nuclear mass. Here, besides $\alpha$ for QED loops and $Z\alpha$ for interactions with the nuclear potential, there is the additional expansion parameter $m_e/M$, where $m_e$ is the electron mass and $M$ is the nuclear mass. To order $\alpha(Z\alpha)^2(m_e/M)^2$, results can be found in Refs.~\cite{eides2010} and to all orders in $m_e/M$ and first order in $Z\alpha$ in Ref.~\cite{pachucki2010}. Suitable for heavy ions, the recoil correction to first order in $m_e/M$, but all orders in $Z\alpha$, is given in Ref.~\cite{shabaev2001,Shabaev2002}.
%introduce nuclear shape and motivate work in this thesis
In Ref.~\cite{jacek2012}, the nuclear shape effect, also called nuclear deformation effect, was introduced for spinless nuclei, which takes a deformed nuclear charge distribution into account. It is not important for light nuclei, but scales strongly with the nuclear charge and therefore becomes important for high $Z$. 

%outlook/outline
The combination of theory and experiment for the bound electron $g$ factor in $^{12}_{\phantom{0}6}$C provided an improved value of the electron mass~\cite{Sturm2014,Zatorski2017}. For $^{28}_{14}$Si~\cite{Sturm2011}, it was shown as a proof-of-principle that nuclear parameters like the RMS charge radius can be obtained. Extracting nuclear magnetic moments was suggested theoretically~\cite{Yerokhin2011,Werth2001}. Also, it was argued that an independent and more accurate value for the fine-structure constant can be obtained~\cite{Shabaev2006,yerokhin2016}. With the upcoming experiments in the high-$Z$ regime, further tests of QED in stong fields, information on nuclear parameters, and extraction of fundamental constants can be expected, and updated theoretical calculations especially for heavy nuclei are needed.
\subsubsection*{Muonic atoms}
For further studies of H-like systems, either the atomic nucleus or the bound electron can be exchanged for another charged particle. This establishes the field of research on \textit{exotic atoms}. Depending on the type of considered particles, nuclear structure effects can either be avoided or enhanced. Bound states between two leptons are not affected by the strong interaction or nuclear effects and are therefore suitable for testing bound state QED. 
% positronium
One example is positronium~\cite{karshenboim2004}, a bound state between an electron and its antiparticle, the positron. The energy levels were studied up to order $\alpha^6$~\cite{pineda1998,pachucki1998,czarnecki1999,zatorski2008} and measured on a $10^{-4}$ up to $10^{-9}$ level, e.g.~\cite{mills1975,ritter1984,danzmann1989,hagena1993,fee1993}, despite a life time in the range of $10^{-9}$s due to pair annihilation. 
% muonium
Another interesting leptonic system is muonium~\cite{jungmann2004}, a bound state consisting of an antimuon and an electron. The muon is the charged lepton in the second generation of matter in the Standard Model. Since the mass ratio of muon and proton is $m_p/m_\mu \approx 8.9$, the spectrum of muonium is quite similar to hydrogen, except that effects due to proton structure are avoided. The hyperfine splitting in muonium was measured to a part-per-billion level in ~\cite{casperson1975,liu1999} and calculated with a similar accuracy~\cite{pachucki1996,Karshenboim1996,sapirstein1997,nio1997,eides1998}. The \textit{MuSEUM} collaboration plans new precision experiments on the hyperfine structure in muonium~\cite{museum}. 
% true muonium / dimuonium
The muon-antimuon bound state is called true muonium~\cite{brodsky2009}, or dimuonium. It has yet to be observed, which is the aim of the $\mu\mu$tron collider, which is constructed at the Budker Institute of Nuclear Physics~\cite{bogomyagkov2017}. 
% true tauonium
A similar bound state exists in principle also for third-generation matter and a tau-antitau bound state is called true tauonium. However, due to the extremely short life time of a tau lepton, it is even more difficult to observe~\cite{brodsky2009}.


%muonic atoms
The muon can also form bound states with atomic nuclei. These exotic atoms are commonly referred to as muonic atoms. In this case, the nuclear structure effects are very important for the following reason: The muon is about 207 times heavier than the electron. Therefore, the Bohr radius of muonic atoms, which is an estimate for the distance between bound particle and nucleus, is also 207 times smaller. As a consequence, especially for high $Z$, the bound muon has a big overlap with the nucleus and it comes to interesting interplays between atomic and nuclear physics. X-rays from muonic atoms were reported for the first time in Ref.~\cite{chang1949} in cosmic ray studies and in Ref.~\cite{fitch1953} with a laboratory muon beam. The beginning of muonic atoms theory is the seminal paper by Wheeler~\cite{wheeler1949}.  Since then, the spectra of muonic atoms have been investigated in numerous experiments, e.g.~\cite{hitlin1970,zehnder1975,powers1976,Yamazaki1978,tanaka1983,tanaka1984,tanaka1984_2,Bergem1988,
powers1977}. In particular, absolute RMS charge radii of atomic nuclei can be obtained by analysis of muonic x-ray spectra~\cite{FRICKE1995}. An overview over the progress in theoretical calculations can be found in Refs.~\cite{BorieRinker1982,Devons1995,wu1969}. In the past, the codes \texttt{MUON} and \texttt{RURP}~\cite{rinker1979} have been used frequently for the comparison of theoretical predictions and measurements of muonic x-rays. 

Muonic hydrogen came to attention recently because laser spectroscopy of a \mbox{\small{$2p\rightarrow 2s$}} transition enabled the extraction of the proton charge radius as $r_p^{(\mu)}=0.84184(67)\,\rm{fm}$ and the result was smaller than CODATA value $0.8768(69)\,\rm{fm}$ at the time by five standard deviations~\cite{Pohl2010}. In 2013, a new measurement increased the deviation to seven standard deviations~\cite{antognini2013}. Measurements on muonic deuterium confirmed these results~\cite{pohl2016} for the deuterium charge radius and lately, also an experiment with atomic hydrogen measured the small proton radius~\cite{beyer2017}.
Results from elastic electron-proton scattering, which are also used for the CODATA value~\cite{CODATA2014}, seem to confirm the large value for the proton radius, but the proton-radius extraction from scattering data is not unambiguous~\cite{arrington2015}. However, a new result from atomic hydrogen spectroscopy resulted in the large value again~\cite{fleurbaey2018} and the so-called \textit{proton radius puzzle} is not resolved until now. Together with anomalies on the magnetic moment of the muon~\cite{bennett2006}, this motivates further investigation of muonic systems.


\subsubsection*{Content of this thesis}
%g factor
To date, the nuclear deformation correction was calculated with perturbative methods only. This thesis contributes to the theory landscape by investigating the nuclear deformation correction non-perturbatively with numerical methods.

%muonic atoms
In the high-$Z$ regime, the MuX collaboration has started an experimental campaign on x-ray spectroscopy of heavy muonic atoms, where the measurements are performed at the Paul Scherrer Institut. 
The aim is to measure muonic x-ray spectra for the heaviest and for the first time also radioactive nuclei, for example $^{226}_{\phantom{1}88}$Ra and $^{248}_{\phantom{1}96}$Cm. 
Therefore, the calculation of muonic spectra is necessary in connection with these experiments. This thesis presents calculations of muonic spectra, using up-to-date numerical schemes and new methods for predicting the higher order hyperfine structure, including vacuum polarization corrections due to deformed nuclei.

\subsubsection*{Structure of the thesis}

This thesis is organized in the following way:\\
Chapter 1 gives an introduction to the framework of bound state QED. The usage of the Dirac equation is discussed, together with analytical and numerical solutions for nuclear potentials and vacuum polarization potentials.\\
In Chapter 2, the nuclear finite size and deformation correction to the bound electron $g$ factor is introduced. A numerical methods for its precise calculation is introduced and compared with the previous results.\\
Chapter 3 is about the energy levels and transition probabilities of muonic atoms. In this connection, new methods and calculations for vacuum polarization and higher-order hyperfine splitting are presented as well as an updated numerical spectrum generator for the extraction of nuclear parameters. In connection with measurements on isotopically pure muonic Rhenium, the nuclear spectroscopic quadrupole moment is extracted.\\
Finally, the main findings of the thesis are summarized in the Summary \& Outlook.




































%\clearpage
%
%
%
%\begin{itemize}
%\item a bit more history for free and bound el. g factor experiment (eg. v. Dyck., Dehmelt)
%\item cite w.q.'s papers
%\item problems with 1-particle interpretation of the Dirac eq. resolved due to QED point of view
%\item connection to nuclear deformation and this thesis
%\item muonic atoms: one possibility for measuring absolute charge radii
%\item for outlook: generalization of nuclear shape correction to non-zero spins
%\end{itemize}
%[heavy ions]\\[11pt]
%[exotic atoms]\\[11pt]
%
%\subsubsection*{Spectroscopy of atoms, picture of atoms in general}
%\begin{itemize}
%\item
%review picture of atom and structure of atom (beginning with rutherford's experiment)\\
%review history of spectroscopy of atoms and connection to development of theories\\
%(discrete spectrum -> quantum mechanics; finestructure -> dirac eq.; lambshift,hfs -> finite nuclear size, qed). \\
%Also mention recent development (H precision spectroscopy, highly charged ions, x ray spectr., microwave spectr.)\\
%\end{itemize}
%
%\subsubsection*{contemporary research}
%\begin{itemize}
%\item recent (precision) spectroscopy
%\item especially heavy ions, nuclear structure effects in spectra
%\end{itemize}
%
%\subsubsection*{bound electron g factor}
%\begin{itemize}
%\item
%history of measurements and theory (Zeeman). recent project (heaviest highly charged ions). penning traps, fine structure constant, electron mass, specific difference etc.
%\end{itemize}
%
%\subsubsection*{muonic atoms}
%\begin{itemize}
%\item
%explain field of exotic atoms (positronium, muonium, true muonium, pionic atoms, muonic atoms) and which regimes can be tested with them. Give more details about muonic atoms, history of findings and results. Describe the new experiments at psi.
%\item 
%explain proton radius puzzle as a motivation for more experiments on muonic systems, also g-2 anomlies for muon
%\end{itemize}





