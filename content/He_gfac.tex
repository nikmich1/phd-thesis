\chapter{Bound muon $\maybebm{g}$ factor in $^4_2\text{He}$}
\label{sec:muon_he}

Another application of the calculations described in this thesis is connected to the bound muon $g$ factor in muonic $^4_2$He, following the work presented in~\cite{sikora2018} by the first author B.~Sikora. Although the helium nucleus has a low charge number, finite nuclear size corrections have to be considered for precise theoretical predictions. In this thesis, the finite nuclear size, electronic Uehling, muonic Uehling, electronic second/higher order Uehling and electronic Källén-Sabry corrections to the bound muon $g$ factor in muonic $^4_2$He were calculated. All effects take an extended nuclear charge distribution into account, and the uncertainty due to the value of the RMS charge radius and model dependence of the nuclear charge distribution is taken into account. 
Other effects, like nuclear polarization, further one- and two-loop QED, recoil, hadronic and weak corrections have been calculated by the other authors in~\cite{sikora2018}. Thereby, a theoretical prediction of the bound-muon $g$ factor in $^4_2$He on the $10^{-9}$ level is obtained.

The calculations are preformed analogously to Chapter~\ref{ch:nucl_def}, but now with a bound muon instead of a bound electron. That is, the Dirac equation
\begin{equation}
\label{eq:boundMuonDirac}
\left[\boldsymbol{\alpha}\cdot\mathbf{p}+
\beta m_\mu + V_i(r)\right]
\,\left|n\kappa m\right> = E_{n\kappa} \, \left|n\kappa m\right>
\end{equation}
is solved for spherically symmetric potentials $V_i(r)$, which are described below. Then, according to Eq.~\eqref{eq:gfac_central}, the $g$ factors $g_i$, including the corrections due to $V_i$ can be obtained by radial integration of the solutions as
\begin{equation}
g_i=\frac{2m_\mu\kappa}{j(j+1)}\int_0^\infty\mathrm{d}r\, r^3 f_{n\kappa}(r)g_{n\kappa}(r).
\label{eq:gfac_centralMuon}
\end{equation}
The finite nuclear size, the electric-loop (Fig.~\ref{fig:vac_pol_uehl} with an external muon and internal electron) and muonic-loop Uehling (Fig.~\ref{fig:vac_pol_uehl} with an external muon and internal muon) correction as well as the Källen-Sabry correction (Fig.~\ref{fig:vac_pol_ks} with an external muon and internal electron) to the bound muon $g$ factor are considered by including the corresponding potentials directly in the Dirac equation. A two-parameter Fermi charge distribution 
\begin{equation}
\rho(r) = \frac{N}{1+\exp\left[(r-c)/a\right]}
\end{equation}
is used, such that the RMS value of $1.6755\,$fm agrees with the literature value from Ref.~\cite{Angeli2013}
The uncertainty of this charge distribution is estimated by using the uncertainty in the RMS value and the model dependence is estimated conventionally by varying the parameters $a$ between $0.05\,$fm and $0.3\,$fm. The considered potentials are the point-like Coulomb potential $V_C(r)$ from Eq.~\eqref{eq:pureCoulomb}, finite size electric potential $V(r)$ from Eq.~\eqref{eq:FSpot}, and Uehling potentials $V^{(m_e)}_{\text{Uehl}}(r)$, $V^{(m_\mu)}_{\text{Uehl}}(r)$ from Eq.~\eqref{eq:uehlPot} for the electric- and muonic-loop Uehling potential, respectively. Furthermore, the Källen-Sabry potential with electronic loops $V^{(m_e)}_{\text{KS}}(r)$ from Eq.~\eqref{eq:KSPot} is taken into account. The $g$-factor corrections are obtained as follows:\\[10pt]
\begin{tabular}{llll}
$i$&potential& $g_i$ factor& $\delta g_{i}/10^{-8}$ correction\\\hline
0&$V_0(r)=V_C(r)$&$1.999\,857\,988\,825\,369$&$\phantom{-1}$--\\
1&$V_1(r)=V(r)$&$1.999\,858\,083\,413\,814$&$+\phantom{1}9.46(4)$\\
2&$V_2(r)=V(r)+V^{(m_{e})}_{\text{Uehl}}(r)$&$1.999\,857\,602\,755\,145$&$-48.0659(4)$\\
3&$V_3(r)=V(r)+V^{(m_{e})}_{\text{Uehl}}(r)+V^{(m_{\mu})}_{\text{Uehl}}(r)$&$1.999\,857\,602\,647\,854$&$-\phantom{1}0.01073(2)$\\
4&$V_4(r)=V(r)+V^{(m_{e})}_{\text{Uehl}}(r)+V_{\text{KS}}(r)$&$1.999\,857\,599\,294\,144$&$-\phantom{1}0.346(1)$\\
\end{tabular}\\[20pt]
The corrections $\delta g_i$ are defined as:\\[10pt]
\begin{tabular}{lll}
correction & definition & effect\\\hline
$\delta g_1$ & $g_1-g_0$ & finite nuclear size correction\\
$\delta g_2$ & $g_2-g_1$ & electronic-loop Ueling correction\\
$\delta g_3$ & $g_3-g_2$ & muonic-loop Ueling correction\\
$\delta g_4$ & $g_4-g_2$ & Källen-Sabry correction\\
\end{tabular}\\[30pt]
Thus, mixed Källen-Sabry and muonic-loop Uehling terms are not considered, but since the individual contributions are already small, the combined contribution is expected to be even smaller and not visible on the $10^{-10}$-level at all. Finally, the electronic-loop Uehling correction can be written as $\delta g_1=47.9600\times 10^{-8}+0.1059\times 10^{-8}$, where the first terms corresponds to the first order Uehling correction, which is the expectation value of the Uehling potential, corresponding to diagram Fig.~\ref{fig:vac_pol_uehl}~(a). The second term corresponds to the second and higher-order Uehling corrections, mainly corresponding to diagram Fig.~\ref{fig:vac_pol_uehl}~(b), but also higher-order diagrams like Fig.~\ref{fig:vac_pol_uehl}~(c). Higher order iterations do not contribute on the $10^{-10}$ level. All calculated contributions to the theoretical prediction of the bound muon $g$ factor in $^4_2$He are presented in Table~\ref{tab:gHe}, where the contributions calculated in this thesis are highlighted in red. There are still uncalculated two-loop light-by-light-scattering diagrams, and the corresponding uncertainty is estimated as $5\times 10^{-9}$~\cite{sikora2018}.
\clearpage
In conclusion, it was demonstrated that the bound-muon $g$ factor in $^4_2$He can be calculated on the $10^{-9}$ level. A measurement of this $g$ factor as~$g_{\text{exp}}$ with a similar experimental accuracy could give access to an independent determination of the muon mass, one order of magnitude more accurate than the current value. 
For this, the dependency of the experimental value $g_{\text{exp}}$ and the theoretical value $g_{\text{theory}}$ on the muon mass has to be solved for an expression of the muon mass in dependency of the experimental and theoretical value as\\
\begin{alignat}{2}
&&&g_{\text{theory}}(m_\mu)\overset{!}{=}g_{\text{exp}}(m_\mu)\\
&\rightarrow &&m_\mu=m_\mu (g_{\text{theory}},g_{\text{exp}}).
\end{alignat}\\
 Alternatively, an independent determination of the muon magnetic moment anomaly of the free muon $g_{\text{free}}-2$ may be possible by separating the  contributions to the free $g$ factor and the binding corrections as $g_{\text{theory}}=g_{\text{free}}+g_{\text{binding}}\overset{!}{=}g_{\text{exp}}$~\cite{sikora2018}.
However, it is important to keep in mind that the life time of the muon is around one microsecond, which is too short-lived for measuring the $g$ factor of muonic atoms in the same way as for electronic atoms, for example in Refs.~\cite{Sturm2011,Sturm2014} and thus a measurement of the bound-muon $g$ factor on the $10^{-9}$ level represents a major experimental challenge.
%
%
\begin{table}
\setlength\extrarowheight{7pt}
%\begin{ruledtabular}
\caption{\label{tab:gHe}Various contributions to the $g$ factor of $\mu{}^4$He$^+$. {eVP}/{$\mu$VP} stands for VP due to virtual $e^-e^+$/$\mu^- \mu^+$ pairs. The estimated uncertainty of the nuclear size
effect stems from the error bar of the nuclear RMS radius and the uncertainty of the nuclear charge distribution model. If not indicated, the uncertainty is negligible.
In the last row, the uncertainties due to the calculated and uncalculated (two-loop light-by-light) terms are given separately. The table and caption is taken from Ref.~\cite{sikora2018} and the contributions highlighted in red were calculated in the framework of this thesis.
}
\centering
\begin{footnotesize}
\begin{tabular}{llll}
\hline \\[-15pt] \hline
Effect                  & Term                  & Numerical value                 & Ref. \\
\hline
Dirac value             &                       & \phantom{-}1.999 857 988 8      & \cite{breit1928,codata} \\
\textcolor{red}{Finite nuclear size}     &                       & \textcolor{red}{\phantom{-}0.000 000 094 6(4)}   & \cite{Angeli2013} \\
Nuclear pol.    &                       & \phantom{-}0.000 000 000 0(10)  &  \\
One-loop SE             & $(Z \alpha)^0$        & \phantom{-}0.002 322 819 5      & \cite{schwinger1948,codata} \\
                        & all-order binding     & \phantom{-}0.000 000 084 9(10)  & \\
One-loop VP             & \textcolor{red}{$e$VP, Uehling}        & \textcolor{red}{-0.000 000 479 6}                &  \\
                        & $e$VP, magnetic loop  & \phantom{-}0.000 000 127 2(4)   & \\
                        & \textcolor{red}{$\mu$VP, Uehling}      &           \textcolor{red}{-0.000 000 000 1}      & \\
                        & hadronic VP, Uehling  &           -0.000 000 000 1(1)   & \\
Two-loop QED            &  $(Z \alpha)^0$       & \phantom{-}0.000 008 264 4      &  \cite{Peterman57,Sommerfield58} \\
                        & SE-SE, $(Z \alpha)^2$--- $(Z \alpha)^5$ & -0.000 000 000 1& \cite{Eides1997,Czarnecki2000,Pachucki2005,czarnecki2018}\\
                        & S(eVP)E, $(Z \alpha)^2$                 & \phantom{-}0.000 000 000 4& \cite{Peterman57,Sommerfield58,Eides1997,Czarnecki2000}\\
                        & \textcolor{red}{Second-order Uehling}  & \textcolor{red}{-0.000 000 001 1(4)}             & \\
                        & \textcolor{red}{K\"all\'en-Sabry}      & \textcolor{red}{-0.000 000 003 5}                & \\
                        & magnetic loop+Uehling & \phantom{-}0.000 000 000 3      & \\
                        & uncalculated LBL      & \phantom{-}0.000 000 000 0 (50) & \\
$\ge$ Three-loop QED    & $(Z \alpha)^0$        & \phantom{-}0.000 000 610 6      & \cite{Laporta96,aoyama2007,Aoyama12,codata} \\
Nuclear recoil          & $\left(\frac{m_{\mu}}{M}\right)^1$, all orders in $Z \alpha$  & \phantom{-}0.000 006 038 2 &  \cite{Shabaev2002}\\
                        & $\left(\frac{m_{\mu}}{M}\right)^{2+}$, $(Z \alpha)^2$ & -0.000 000 488 7 &  \cite{Pachucki2008}\\
                        & radiative recoil      &  -0.000 000 004 7 & \cite{Grotch1970}\\
Weak interaction        & $(Z\alpha)^0$         &  \phantom{-}0.000 000 003 1     & \cite{Czarnecki96,codata} \\
Hadronic  &  $(Z\alpha)^0$        &  \phantom{-}0.000 000 139 3(12) & \cite{Prades10,Nomura13,Kurz14,codata} \\
\hline\\[-17pt]
Sum of terms calculated &     &  \multicolumn{2}{l}{\phantom{-}2.002 195 193 4(20)${}_{\rm calc}$(50)${}_{\rm uncalc}$}\\
\hline \\[-15pt] \hline
\end{tabular}
%\end{ruledtabular}
\end{footnotesize}
\end{table}
%
%