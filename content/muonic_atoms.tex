\chapter{Hyperfine structure of muonic atoms}
\label{ch:muonic_atoms}
This chapter is devoted to the prediction of the level structure and transition probabilities in muonic atoms with focus on high nuclear charge numbers. Compared to conventional atomic electrons, the much higher muon mass reduces the length- and  increases the energy scales by the muon-electron mass ratio. Thereby, all finite nuclear size and shape effects are much more important and also excited nuclear states have to be taken into account.\\
This chapter is organized in the following way:\\
At first, a motivation for the need for new structure calculations is given in Section~\ref{sec:muon_motivation} by comparing to existing methods and by pointing new experiments in the field out, which use the results presented in this thesis.\\
Afterwards, the theoretical framework and implementation of several known effects is shown in Sections~\ref{sec:muon_framework} to~\ref{sec:muon_dynamic}, using finite basis set methods suitable for precision calculations and presenting results for selected nuclei.\\
In Sections~\ref{sec:muon_residualSO} and~\ref{sec:muon_quadUehl}, improved methods for calculating higher order effects in the electric hyperfine interactions are presented.\\
Finally, in Sections~\ref{sec:muon_re}, calculations in combination with experiments on isotopically pure Rhenium-185 are presented, including the extraction of the nuclear quadrupole moment by combining theoretical predictions and measurements of muonic transitions. \\
In Section~\ref{sec:muon_he}, precision calculations for the vacuum polarization corrections to the bound-muon $g$ factor in muonic Helium-4 are shown.

\section{Calculation of spectra for muonic atoms}
\subsection{Motivation}
\label{sec:muon_motivation}
Atomic nuclei are one of the building blocks of matter and therefore, information on their structure, like the distribution of electric charge inside the nucleus, is of intrinsic interest. Furthermore, the charge radii of atomic nuclei are of importance as an input parameter for the interpretation of other experiments. For example, radium atoms are a candidate for measuring atomic parity violation effects, but for this more accurate values of the radium charge radii are needed~\cite{wansbeek2012}.
There are several methods for extracting information on the nuclear charge distribution, i.e. the distribution of protons inside the atomic nucleus, like electron scattering~\cite{devries1987} or laser spectroscopy~\cite{wang2004,dewitte2007,mueller2007}. One method is also muonic atom spectroscopy. Here, a muon, which is a negatively charged elementary particle is brought in the proximity of an atomic nucleus. Then, the negatively charged muon forms bound states with the positively charged nucleus and radiation due to muonic transitions can be analyzed in order to obtain information on the nuclear charge distribution and measure quantities like nuclear charge radii and quadrupole moments.

Correspondingly, the theory of muonic atoms has been developed in order to describe the level structure and the probabilities for muonic transitions. The general approach is that for a given nuclear charge distribution, the spectrum of the corresponding muonic atom needs to be predicted. Then, vice versa, for a measured spectrum the nuclear charge distribution can be extracted. An overview of the different contributions to the energy levels of muonic atoms can be found in~\cite{BorieRinker1982}. Hitherto, the majority of analyses of the spectra of heavy muonic atoms used the computer programs \textit{MUON} and \textit{RURP}~\cite{rinker1979}. There are two main differences compared to the approach used in this thesis:
Firstly, the dual-kinetic-balance method~\cite{Shabaev2004} is used in this thesis. Being a finite basis set method, a approximation of the complete muon spectrum is obtained, i.e. bound states and positive as well as negative continuum states, including the effects of the finite nuclear charge distribution. Thereby, numerical summations over the complete muon spectrum are possible. This enables for example a complete treatment of the second order hyperfine interactions, as presented in Section~\ref{sec:muon_residualSO}.
Secondly, whereas in the \textit{MUON} and \textit{RURP} codes, the fine and hyperfine structure are calculated separately, in this thesis the calculations of the fine and hyperfine structure are performed at once, based on a given nuclear charge distribution, enabling improved analysis of the dependence of the muonic spectrum on parameters of the nuclear charge distribution.

Furthermore, there are new experiments on high $Z$ muonic atoms being performed by the muX-Collaboration at the Paul-Scherrer-Institute in Villigen, Switzerland~\cite{kirch2016}. One of the goals is to measure the charge radius for Radium, needed for experiments on atomic parity violations, as mentioned earlier in this Section. Furthermore, measurements on muonic atoms involving several nuclei will be or have been performed for the first time, involving isotopically pure Rhenium. The corresponding structure calculations for muonic Rhenium were performed during the work on this thesis, and results are presented in Section~\ref{sec:muon_re}.

\subsection{Theoretical framework}
\label{sec:muon_framework}
A muon is a charged elementary particle, which is in many aspects similar to the electron, in particular, it has the same electric charge, but it is ${\approx}\,{200}$ times heavier than the electron~\cite{codata}. When coming close to an atom, a muon can be captured by the nucleus and form a hydrogen-like muonic ion. This atomic system is commonly referred to as a muonic atom. The lifetime of the muon is big enough to be considered stable in the structure calculations of these muonic bound states.
The reason why muonic atoms, compared to conventional electronic atoms, are suitable for extraction information on nuclear structure is that the muon is about 207 times heavier than the electron. This can be seen by considering the ratio of the nuclear radius and the Bohr radius of the bound fermion, which is the typical length scale for the bound muon or electron. The larger this ratio is, the larger are nuclear finite size effects. The Bohr radius for a hydrogen-like atomic system with a bound fermion of mass $m_f$ and a nuclear charge number $Z$ reads as
\begin{equation}
r_B = \hbar / (m_f c Z \alpha),
\end{equation}
where $\hbar$ is the Planck's constant, $\alpha$ is the finestructure constant and $c$ is the speed of light in vacuum. In Table~\ref{tab:nucl_radii}, the nuclear radius, the Bohr radius for the corresponding electronic and muonic hydrogen-like ion, and their ratio is shown for a selection of nuclei, from very light to very heavy.
%
%
\begin{table}
\caption{\label{tab:nucl_radii}
Comparison of the ratio of nuclear charge radius $R_N$ to the Bohr radius $r_B$ of a bound electron or muon in the corresponding hydrogen-like atomic system for Hydrogen, Helium, Rhenium, and Uranium. If this ratio is small, the finite size of the nucleus does not influence the bound fermion significantly. On the other hand, if this ratio is on the order of 1, large finite nuclear size and nuclear structure effects can be expected. The nuclear charge radii are taken from~\cite{Angeli2013}.}
\centering
\begin{tabular}{ccccc}
Fermion type & Nucleus & $R_N$[fm]&$r_B$[fm]&$R_N/r_B$\\\hline\\[-5pt]
$e^-$&$^1_1$H&0.8783&$52917.721$&$1.660\times 10^{-5}$\\[3pt]
$\mu^-$&$^1_1$H&0.8783&$\phantom{11}255.928$&$3.432\times 10^{-3}$\\[15pt]
$e^-$&$^4_2$He&1.6755&$26458.861$&$6.332\times 10^{-5}$\\[3pt]
$\mu^-$&$^4_2$He&1.6755&$\phantom{11}127.964$&$1.309\times 10^{-2}$\\[15pt]
$e^-$&$^{185}_{\phantom{1}75}$Re&5.3596&$\phantom{11}705.570$&$7.596\times 10^{-3}$\\[3pt]
$\mu^-$&$^{185}_{\phantom{1}75}$Re&5.3596&$\phantom{1111}3.412$&$1.571\phantom{111111}$\\[15pt]
$e^-$&$^{238}_{\phantom{1}92}$U&5.8571&$\phantom{11}575.193$&$1.018\times 10^{-2}$\\[3pt]
$\mu^-$&$^{238}_{\phantom{1}92}$U&5.8571&$\phantom{1111}2.782$&$2.105\phantom{111111}$
\end{tabular}
\end{table}
%
%
It can be seen that, that for electronic atoms, the nucleus is generally a few orders of magnitude smaller compared to the extend of the electronic wave function, which is given by the Bohr radius, although for high $Z$, the electron is much closer to the nucleus due to the strong Coulomb attraction. Since the Bohr radius is inversely proportional to the mass of the bound fermion, the situation in muonic atomic systems is different. While for low $Z$, the extend of the muonic wave functions is still much larger than the nuclear radius, for high $Z$, the nuclear radius is actually larger than the muonic Bohr radius. This means the overlap between muonic wave functions and nucleus is large in this case. Also, a typical energy scale for hydrogen-like systems is the ground state binding energy from Eq.~\eqref{eq:finestructure_formula} for a point-like nucleus, which reads as
\begin{equation}
E_{0,\text{point}}=m_f c^2 (1-\sqrt{1-(Z\alpha)^2}),
\end{equation}
and is proportional the fermion mass. As a consequence, for muonic atoms and high charge numbers, muonic transitions can have an energy of several mega electronvolts and fine structure splitting can be on the order of several hundreds of kilo electronvolts. This is of the same order as excitation energies of nuclear rotational states~\cite{ENSDF} and therefore a nuclear model has to be used, which besides an extended charge distribution also contains the excited nuclear states of the ground state rotational band. Here, the symmetric rigid rotor model model for nuclei with axial symmetry has proved successful in describing heavy muonic atoms, e.g.~\cite{tanaka1984,hitlin1970,wu1969,Devons1995} and also is used in this thesis. The symmetric rigid rotor model is introduced in Appendix~\ref{app:rig_rotor}, where also the expressions for the nuclear wave functions in terms of Wigner D functions can be found. In the symmetric rigid rotor model, the nucleus is described by a charge distribution $\rho(\mathbf{r})$ given in the body fixed nuclear frame, and the Euler angles $\Omega=(\phi,\theta,\psi)$ describe its position in the laboratory frame. The rotational state of the nucleus $\left|IMK\right>$ is given by the total nuclear angular momentum quantum number $I$, its projection on the $z$ axis of the laboratory frame $M$ and on the $z$ axis of the body fixed frame $K$, where $K$ also corresponds to the ground state angular momentum. As derived in Chapter~\ref{ch:furry_pic}, the muonic bound states without radiative corrections can be obtained by solving the Dirac equation for the nuclear potential. The coupled system of muon as a Dirac particle and nucleus as a rigid rotor therefore is described by the eigenvalue equation
\begin{equation}
\text{H} \left(\left|N\right>\otimes \left|\mu\right>\right)= \left(\text{H}_{\text{N}} + \text{H}_\mu + V_{\text{el}}\right) \left(\left|N\right>\otimes \left|\mu\right>\right)= E \left(\left|N\right>\otimes \left|\mu\right>\right),
\label{eq:muon_htotal}
\end{equation}
where $H_{\text{N}}$ is the nuclear rigid rotor Hamiltonian, and ${H_\mu}{=}{\vec{\alpha} \cdot \vec{p} + \beta m_\mu}$ is the free Dirac Hamiltonian for the muon with momentum $\vec{p}$, and $\vec{\alpha}$ and $\beta$ are the four Dirac matrices. $\left|N\right>$ is the nuclear state and $\left|\mu\right>$ is the muon state. Following Eq.~\eqref{eq:furry_elPot}, the electric potential energy between muon and nuclear charge distribution reads as
\begin{equation}
\label{eq:muon_elInteract}
V_{\text{el}}(\mathbf{r}_\mu^\prime)=-Z\alpha \int \mathrm{d}^3\mathbf{r}_N^\prime \frac{\rho(\mathbf{r}_N^\prime)}{|\mathbf{r}_\mu^\prime-\mathbf{r}_N^\prime|}.
\end{equation}
It is important to recall, that the nuclear charge distribution is given in the body fixed nuclear frame, thus the integration in Eq.~\eqref{eq:muon_elInteract} in most conveniently performed in this frame. The resulting expression in dependence of the position of the muon, however, is needed in the laboratory frame. Therefore, in the following, a multipole expansion of Eq.~\eqref{eq:muon_elInteract} is performed, and the result is given as a function of the muon position in the laboratory frame and the Euler angles describing the orientation of the nuclear frame. Here, vectors are written is spherical components as $\mathbf{r}_i = (r_i,\vartheta_i,\varphi_i)$ in the laboratory frame and as $\mathbf{r}_i^\prime = (r_i^\prime,\vartheta_i^\prime,\varphi_i^\prime)$ in the body-fixed nuclear frame. Since rotations do not change the absolute value of rotations, it holds that $r_i^\prime = r_i$.\\

With the multipole expansion of the Coulomb potential~\cite{jackson1999}
\begin{equation}
\frac{1}{|\mathbf{r}_\mu^{\,\prime}-\mathbf{r}_N^{\,\prime}|}=\sum_{l=0}^\infty \frac{r_<^l}{r_>^{l+1}}\sum_{m=-l}^l  \text{C}^*_{lm}(\vartheta_N^{\prime},\varphi_N^{\prime})\text{C}_{lm}(\vartheta_\mu^\prime,\varphi_\mu^\prime),
\end{equation}
where $r_<:=\min (r_\mu,r_N)$, $r_>:=\max (r_\mu,r_N)$, the electric potential~\eqref{eq:muon_elInteract} can be written as
\begin{align}
V_{\text{el}}(\mathbf{r}_\mu^{\,\prime})=&\sum_{l,m}-Z\alpha \left[ \int \text{d}^3r^{\prime}_N \frac{r_<^l}{r_>^{l+1}}\text{C}^*_{lm}(\vartheta_N^{\prime},\varphi_N^{\prime})\rho(\vec{r}_N^{\,\prime})\right]\notag\\
&\,\times\text{C}_{lm}(\vartheta_\mu^{\prime},\varphi_\mu^\prime),
\end{align}
where ${\text{C}_{lm}(\vartheta,\varphi)}{=}{\sqrt{4\pi/(2l+1)}\text{Y}_{lm}(\vartheta,\varphi)}$ are the normalized spherical harmonics.
For axially symmetric charge distributions, only the ${m}{=}{0}$-terms are non-zero after integrating over the charge distribution. The dependency on the muonic angular variables can be transformed to the laboratory system using the Euler angles by
\begin{equation}
\text{P}_{l}(\cos\vartheta_\mu^\prime)=
 \sum_{m=-l}^l C^*_{lm}(\theta,\phi)C_{lm}(\vartheta_\mu,\varphi_\mu),
\end{equation}
which is a special case of Eq.~\eqref{eq:rot_sphHarm}.
Thereby, the potential as a function of the Euler angles and the muon's position in the laboratory frame reads
\begin{align}
V_{\text{el}}(\mathbf{r}_\mu,\phi,\theta)=&\sum_{l=0}^\infty-Z\alpha \left[ \int \text{d}^3r^{\prime}_N \frac{r_<^l}{r_>^{l+1}}\text{P}_{l}(\cos \vartheta_N^{\prime})\rho(\vec{r}_N^{\,\prime})\right]\sum_{m=-l}^l \text{C}^*_{lm}(\theta,\phi)\text{C}_{lm}(\vartheta_\mu,\varphi_\mu).\notag\\
\eqqcolon &\sum_{l=0}^\infty Q_{\text{el}}^{(l)}(r_\mu)\sum_{m=-l}^lC^*_{lm}(\theta,\phi)C_{lm}(\vartheta_\mu,\varphi_\mu)&\notag\\
\eqqcolon& \sum_{l=0}^\infty V^{(l)}_{\text{el}}(\mathbf{r}_\mu,\phi,\theta),
\label{eq:defmulti}
\end{align}
where $Q_{\text{el}}^{(l)}(r_\mu)$ describe the radial distribution of the multipole moments and the dependency on the muonic angles and the Euler angles is in form of a scalar product of spherical tensor operators. This means, as expected the interaction energy does only depend on the relative orientation of the muon with respect to the body fixed nuclear $z^\prime$ axis.

Most nuclei turn out to be symmetric with respect to reflection in the $(x^\prime ,y^\prime)$-plane~\cite{zickendraht1991}. This results in only even-$l$ terms in Eq.~\eqref{eq:defmulti} being non-zero, thus the first to terms are the monopole term $l=0$ and the quadrupole term $l=2$. The next term would be the $l=4$ hexadecapole terms, however, usually, higher order terms are not needed for the correct description of the hyperfine level structure. It follows from Eq.~\eqref{eq:defmulti}, that the monopole terms only depends on the muonic radial variable. In fact, the $l=0$ terms is the averaged nuclear potential already derived in Eq.~\eqref{eq:gfac_monopole} in the previous chapter on the $g$ factor of spinless nuclei. As a consequence, the $l=0$-term can be used as a potential in the spherically symmetric Dirac equation to define the unperturbed muonic states as
\begin{equation}
\label{eq:muonicEn}
\left(\boldsymbol{\alpha} \cdot \mathbf{p} + \beta m_\mu + V_{\text{el}}^{(0)}(r_\mu) \right) \left|n\kappa m\right> = E_{n\kappa}\left|n\kappa m\right>.
\end{equation}
The unperturbed nuclear states are the rigid rotor states from Eq.~\eqref{app:rigidRot_state} with
\begin{equation}
\label{eq:nuclEn}
\text{H}_N \left|IMK\right> = E_N \left|IMK\right>,
\end{equation}
where the excitation energies of the nuclear rotational states are typically taken from experiment~\cite{ENSDF}, and not parametrized by the moment of inertia from Eq.~\eqref{eq:rig_rotorEn}.
The $l=2$ quadrupole term couples nuclear and muonic degrees of freedom and is treated in perturbation theory in Sections~\ref{sec:elQuad1},~\ref{sec:muon_dynamic}, and~\ref{sec:muon_residualSO}. The multipole interaction from Eq.~\eqref{eq:defmulti} in general, and the quadrupole interaction with $l=2$ in particular has the structure of a scalar product of two irreducible tensor operators, as defined in Eq.~\eqref{app:tensor_scalarProd}. One operator acts on the nuclear angular variables, i.e. the Euler angles, and one on the muonic angular variables.
For these types of operators, the calculation of expectation values can be simplified by using the theory of irreducible tensor operators, as described in Appendix~\ref{app:ang_theo}. For applying formula~\eqref{app:reducedMatEl_expectation}. Therefore, states of defined total angular momentum have to be considered as
\begin{equation}
\left| FM\,IKj(\kappa)\right> = \sum_{M_N, m}\text{C}^{FM}_{IM_N\,jm}\left|IM_NK\right>\,\left|n\kappa m\right>
\end{equation}
with total angular momentum $F$, nuclear angular momentum $I$ and muonic angular momentum $j$ are defined. The muonic angular momentum $j$ is already composed out of the orbital angular momentum $l$ and the spin angular momentum, as described in Eq.~\eqref{eq:ansatz_dirac}. The total energy of the muon-nucleus system can be calculated as the sum of nuclear energy $E_N$ from Eq.~\eqref{eq:nuclEn}, the muonic energy from Eq.~\eqref{eq:muonicEn} and the expactation value of the quadrupole interaction $\left<V_{\text{el}}^{(2)}\right>$ from Eq.~\eqref{eq:defmulti}.

\subsection{Fine and first order hyperfine structure}
\label{sec:muon_finestructure}
In this Section, the solution of the Dirac equation for the muon will be discussed, in particular the effects of vacuum polarization in Uehling approximation, recoil corrections, electron screening are implemented, using known methods. Especially, the usage of the dual-kinetic-balance method~\cite{Shabaev2004} in the framework of muonic atoms is presented. In the following the first order hyperfine structure is considered. Results are presented for muonic $^{205}$Bi, $^{147}$Sm, and $^{89}$Zr, following~\cite{michel2017}, which was published in the framework of this thesis.

\subsubsection{Finite nuclear size corrections}
\label{sec:radialEq}
The total Hamiltonian for a muon bound to a nucleus can be written as a sum of nuclear, muonic, and interaction Hamiltonian~\cite{Devons1995}. Thus, we consider the Hamiltonian
\begin{equation}
H = H_{N} + H^{(0)}_{\mu} + H_{\mu - N},
\end{equation}
with the nuclear Hamiltonian $H_{N}$, the Dirac Hamiltonian $H^{(0)}_{\mu}$ for the free muon, and the interaction Hamiltonian $H_{\mu - N}$. The nucleus is described in the rotational model, i.e. in a state with well defined angular momentum and charge- and current density in the body fixed nuclear frame~\cite{kozhedub2008}. As a next step, the interaction between the bound muon and the atomic nucleus is expanded, where electric and magnetic interactions are taken into account. The interaction Hamiltonian is
\begin{equation}
\label{eq:Hint}
H_{\mu - N} = H_{E} + H_{M}
\end{equation}
where the electric part reads
\begin{equation}
\label{eq:elInt}
H_{E}= - \alpha \int \mathrm{d}V^{\prime}\, \frac{\rho (\vec{r}^{\,\prime})}{|\vec{r}_{\mu}-\vec{r}^{\,\prime}|} ,
\end{equation}
with the fine structure constant $\alpha$, the position $\vec{r}^{\,\prime}$ of the nuclear charge distribution and the position $\vec{r}_{\mu}^{\,\prime}$ of the muon in the nuclear frame. The nuclear charge distribution $\rho(\vec{r})$ is normalized to the nuclear charge $Z$ as
\begin{equation}
\label{eq:norm}
\int \mathrm{d}V\rho(\vec{r}) = Z.
\end{equation}
Conveniently, the nuclear charge distribution is divided into a spherically symmetric part $\rho_0(r)$ and a part $\rho_2(r)$ describing the quadrupole distribution in the nuclear frame as~\cite{hitlin1970}
\begin{equation}
\label{eq:rho}
\rho(\vec{r}^{\,\prime}) = \rho_0(r^{\prime}) + \rho_2(r^{\prime}) \, Y_{20}(\vartheta^\prime,\varphi^\prime),
\end{equation}
with the spherical harmonics $Y_{lm}(\vartheta,\varphi)$. Since an analogous part for the dipole distribution would be an operator of odd parity, it would vanish after averaging with muon wave functions of defined parity~\cite{johnson2007}, and thus it is not considered here and neither are higher multipoles beyond the quadrupole term. Correspondingly, the electric interaction Hamiltonian from~\eqref{eq:Hint} can be written as
\begin{equation}
\label{eq:quadInt}
H_E = H^{(0)}_E + H^{(2)}_E,
\end{equation}
where the spherically symmetric part of the charge distribution gives rise to
\begin{equation}
\label{eq:Hmonopole}
H^{(0)}_E(r_\mu)= - 4 \pi \alpha \int_0^\infty \mathrm{d}r \, r^2 \frac{\rho_0(r)}{r_>},
\end{equation}
with $r_>=\text{max}(r,r_\mu)$. This interaction Hamiltonian will be included in the numerical solution of the Dirac equation for the muon as described in Sec.~\ref{sec:radialEq}. The quadrupole part of the interaction $H^{(2)}_E$ causes hyperfine splitting, which is calculated perturbatively in Sec.~\ref{sec:elQuad1}.\\

For evaluating these Hamiltonians, the appropriate states are states of defined total angular momentum. A nuclear state $\ket{IM}$ with nuclear angular momentum quantum number $I$ and projection $M$ on the $z$ axis of the laboratory frame and a muonic state $\ket{n\kappa m}$ with total angular momentum $j(\kappa)=|\kappa|-\frac{1}{2}$ and projection $m$ are coupled to a state $\ket{FM_FI\kappa}$ with angular momentum $F$ and projection $M_F$ as
\begin{equation}
\label{eq:totalState}
\ket{FM_FI\kappa}=\sum_{M,m} C^{FM_F}_{IM\,jm} \ket{I M} \, \ket{n\kappa m},
\end{equation}
where $C^{jm}_{j_1m_1j_2m_2}$ are the Clebsch-Gordan coefficients~\cite{varshalovich1988}. Here, $n$ is the principal quantum number of the muon and $\kappa=(-1)^{j+l+\frac{1}{2}}(j+\frac{1}{2})$ with the orbital angular momentum quantum number $l$.

As a basis for further calculations, the Dirac equation
\begin{equation}
\label{eq:diracSph}
\left( \vec{\alpha}\cdot \vec{p}+ \beta + V(r_\mu) \right) \ket{n \kappa m} = (1-E_{n \kappa}) \ket{n \kappa m}
\end{equation}
is solved for the muon. Here, $\vec{\alpha}$ and $\beta$ are the four Dirac matrices, $E_{n \kappa}$ are the binding energies, and the potential $V(r)$ is the spherically symmetric part of the interaction with the nucleus, which is the monopole contribution from the electric interaction in Eq.~\eqref{eq:Hmonopole} and the Uehling potential from Eq.~\eqref{eq:uehl_2}. A Fermi type charge distribution~\cite{Beier2000} is used to model the monopole charge distribution as
\begin{equation}
\label{eq:fermi}
\rho_0 (r)=\frac{N}{1+\text{exp}((r-c)/a)},
\end{equation}
where $a$ is a skin thickness parameter and $c$ the half-density radius. The normalization constant $N$ is chosen such that the volume integral is equal to one, since the charge is already included in the fine-structure constant. It has been proven, that $a=t/(4\,\text{log}3)$, with $t=2.30\,\text{fm}$, is a good approximation for most of the nuclei~\cite{Beier2000}. The parameter $c$ is then determined by demanding, that the charge radius squared
\begin{equation}
\left<r^2\right>=\cfrac{\int\text{d}r \, r^4\rho_0(r)}{\int\text{d}r \, r^2\rho_0(r)}
\end{equation}
agrees with the values from the literature~\cite{Angeli2013}. Since the potential in Eq.~\eqref{eq:diracSph} is spherically symmetric, the angular part can be separated and the solution with spherical spinors $\Omega_{\kappa m}(\vartheta,\varphi)$ can be written as~\cite{greiner2000}
\begin{equation}
\ket{n\kappa m}=\frac{1}{r}\colvec{2}{G_{n\kappa}(r)\,\Omega_{\kappa m}}{i\,F_{n\kappa}(r)\,\Omega_{-\kappa m}},
\end{equation}
and the resulting equations for the radial functions are solved with the dual-kinetic-balance method~\cite{Shabaev2004} to obtain $G_{n\kappa}$ and $F_{n\kappa}$, and the corresponding eigenenergies numerically.

In Table~\ref{tab:sphDirac}, the binding energies for muonic $^{205}_{83}$Bi, $^{147}_{62}$Sm, and $^{89}_{40}$Zr are shown, both with and without the corrections from the Uehling potential in Eq.~\eqref{eq:uehl_2}. The finite nuclear size effect is illustrated by also including the binding energies $E^{(C)}_{n\kappa}$ of the pure Coulomb potential $-Z\alpha / r_\mu$, which read~\cite{greiner2000}
\begin{equation}
\label{eq:finestructure_formula_2}
E^{(C)}_{n\kappa}=1-\left(1+\frac{(Z\alpha)^2}{\left( n-|\kappa|+\sqrt{\kappa^2-(Z\alpha)^2} \right)^2}\right)^{-\frac{1}{2}}.
\end{equation}
The uncertainties include the error in the rms radius value as well as a model error, which is estimated via the difference of the binding energies with the Fermi potential from Eq.~\eqref{eq:fermi} and the potential of a charged sphere with the same rms radius. For heavy nuclei, the finite nuclear size correction can amount up to 50$\,\%$, and thus the binding energy is halved.

%begin table
\begin{table}[b]
\caption{\label{tab:sphDirac}
Overview of the binding energies for muonic $^{205}_{83}$Bi, $^{147}_{62}$Sm, and $^{89}_{40}$Zr, obtained by solving the Dirac equation with the spherically symmetric parts of the muon-nucleus interaction. The values for solving the Dirac equation only with the electric monopole potential, and with the electric monopole potential and the Uehling potential are presented to show the influence of the leading order vacuum polarization. The binding energies \eqref{eq:finestructure_formula_2} for a  point like nucleus are shown as well. The reduced mass is used to include the non-relativistic recoil corrections from Section~\ref{sec:recoil}. The corrections from section~\ref{sec:screen} are not included in this table. All energies are in keV.}
%\begin{ruledtabular}
\centering
\begin{tabular}{cclll}
& state & \text{point like}& \text{finite size (fs)}\footnotemark[1] &\text{fs+Uehling}\footnotemark[2]\\ \hline \\[-7pt]
$^{205}$Bi & 1s\nicefrac{1}{2} &\text{21573.3} & \text{10699.(51.)} &\text{10767.(52.)} \\
  & 2s\nicefrac{1}{2} & \text{\phantom{1}5538.6} & \text{\phantom{1}3654.(15.)} & \text{\phantom{1}3674.(15.)}\\
  & 2p\nicefrac{1}{2} & \text{\phantom{1}5538.6} & \text{\phantom{1}4893.(3.)} & \text{\phantom{1}4927.(3.)} \\
  & 2p\nicefrac{3}{2} & \text{\phantom{1}4958.9} & \text{\phantom{1}4706.(5.)} & \text{\phantom{1}4737.(5.)} \\
  & 3s\nicefrac{1}{2} & \text{\phantom{1}2394.3} & \text{\phantom{1}1796.(5.)} & \text{\phantom{1}1804.(6.)} \\
  & 3p\nicefrac{1}{2} & \text{\phantom{1}2394.3} & \text{\phantom{1}2170.0(5)} & \text{\phantom{1}2190.1(5)} \\
  & 3p\nicefrac{3}{2} & \text{\phantom{1}2221.4} & \text{\phantom{1}2131.(1.)} & \text{\phantom{1}2141.(1.)} \\
  & 3d\nicefrac{3}{2} & \text{\phantom{1}2221.4} & \text{\phantom{1}2216.9(3)}& \text{\phantom{1}2227.8(3)}\\
  & 3d\nicefrac{5}{2} & \text{\phantom{1}2174.6} & \text{\phantom{1}2172.8(2)} & \text{\phantom{1}2183.0(2)} \\[7pt]
 $^{147}$Sm & 1s\nicefrac{1}{2} & \text{11423.8} & \text{\phantom{1}7165.(28.)} & \text{\phantom{1}7213.(29.)} \\
  & 2s\nicefrac{1}{2} & \text{\phantom{1}2895.7} & \text{\phantom{1}2230.(7.)} & \text{\phantom{1}2242.(7.)} \\
  & 2p\nicefrac{1}{2} & \text{\phantom{1}2895.7} & \text{\phantom{1}2778.(2.)} & \text{\phantom{1}2795.(2.)} \\
  & 2p\nicefrac{3}{2} & \text{\phantom{1}2736.9} & \text{\phantom{1}2689.(2.)} & \text{\phantom{1}2706.(2.)} \\
  & 3s\nicefrac{1}{2} & \text{\phantom{1}1268.9} & \text{\phantom{1}1061.(2.)} & \text{\phantom{1}1066.(2.)} \\
  & 3p\nicefrac{1}{2} & \text{\phantom{1}1268.9} & \text{\phantom{1}1228.6(4)} & \text{\phantom{1}1234.2(4)} \\
  & 3p\nicefrac{3}{2} & \text{\phantom{1}1221.7} & \text{\phantom{1}1204.7(6)} & \text{\phantom{1}1210.0(6)} \\
  & 3d\nicefrac{3}{2} & \text{\phantom{1}1221.7} & \text{\phantom{1}1221.4(1)} & \text{\phantom{1}1226.2(1)} \\
  & 3d\nicefrac{5}{2} & \text{\phantom{1}1207.6} & \text{\phantom{1}1207.4} & \text{\phantom{1}1212.1} \\[7pt]
 $^{89}$Zr & 1s\nicefrac{1}{2} & \text{\phantom{1}4595.5} & \text{\phantom{1}3643.(8.)} & \text{\phantom{1}3669.(8.)} \\
  & 2s\nicefrac{1}{2} & \text{\phantom{1}1155.2} & \text{\phantom{1}1021.(2.)} & \text{\phantom{1}1026.(2.)} \\
  & 2p\nicefrac{1}{2} & \text{\phantom{1}1155.2} & \text{\phantom{1}1147.8(2)} & \text{\phantom{1}1153.7(2)} \\
  & 2p\nicefrac{3}{2} & \text{\phantom{1}1129.9} & \text{\phantom{1}1127.0(2)} & \text{\phantom{1}1132.6(2)} \\
  & 3s\nicefrac{1}{2} & \text{\phantom{11}510.6} & \text{\phantom{11}469.8(5)} & \text{\phantom{11}471.4(5)} \\
  & 3p\nicefrac{1}{2} & \text{\phantom{11}510.6} & \text{\phantom{11}508.0(1)} & \text{\phantom{11}509.8(1)} \\
  & 3p\nicefrac{3}{2} & \text{\phantom{11}503.1} & \text{\phantom{11}502.0(1)} & \text{\phantom{11}503.8(1)} \\
  & 3d\nicefrac{3}{2} & \text{\phantom{11}503.1} & \text{\phantom{11}503.1} & \text{\phantom{11}504.5} \\
  & 3d\nicefrac{5}{2} & \text{\phantom{11}500.7} & \text{\phantom{11}500.7} & \text{\phantom{11}502.1} \\

\end{tabular}
%\end{ruledtabular}
\footnotetext[1]{$V(r_\mu)=H^{(0)}_E(r_\mu)$}
\footnotetext[2]{$V(r_\mu)=H^{(0)}_E(r_\mu)+V_{\text{Uehl}}(r_\mu)$
%\footnotetext[3]{$V(r_\mu)=-Z\alpha/r_\mu$
\\see Eq.~\eqref{eq:Hmonopole}, \eqref{eq:diracSph}, and \eqref{eq:uehl_2} for definitions}
\end{table}
%end table
%
%
\subsubsection{Vacuum polarization}
\label{sec:qed}

For atomic electrons, usually the self-energy QED correction is comparable to the vacuum polarization correction~\cite{Beier2000}. For muons, however, the vacuum polarization correction is much larger due to virtual electron-positron pairs, which are less suppressed due to their low mass compared to the muon's mass~\cite{BorieRinker1982}. The spherically symmetric part of the vacuum polarization to first order in $\alpha$ and $Z\alpha$ is the Uehling potential~\cite{Elizarov2005}
\begin{align}
V_{\text{Uehl}}(r_\mu)=-\alpha \frac{2\alpha}{3\pi}\int_0^\infty \text{d}r^{\prime}\,4\pi \rho_0(r^\prime)\int_1^\infty \text{d}t\,\left( 1+\frac{1}{2t^2} \right)\nonumber\\
\times\frac{\sqrt{t^2-1}}{t^2} \frac{\text{exp}(-2m_e|r_\mu-r^\prime|t)-\text{exp}(-2m_e(r_\mu+r^\prime)t)}{4m_er_\mu t},
\label{eq:uehl_2}
\end{align}
where $m_e$ is the electron mass and $\rho_0$ is the spherically symmetric part of the charge distribution from Eq.~\eqref{eq:rho}. This potential can be directly added to the Dirac equation \eqref{eq:diracSph}. In this way, all iterations of the Uehling potential are included~\cite{indelicato2013}. Results for our calculations can be found in Table~\ref{tab:sphDirac}.
%
%
\subsubsection{Recoil corrections}
\label{sec:recoil}
Taking into account the finite mass and the resulting motion of the nucleus leads to recoil corrections to the bound muon energy levels. In nonrelativistic quantum mechanics, as in classical mechanics, the problem of describing two interacting particles can be reduced to a one particle problem by using the reduced mass $m_r$ of the muon-nucleus system~\cite{landaulifshitz3}. With the mass of the nucleus $m_N$, the reduced mass reads in the chosen system of units as
\begin{equation}
\label{eq:redmass}
m_r=\frac{m_N}{m_N+1},
\end{equation}
and the Dirac equation is accordingly modified to
\begin{equation}
\label{eq:diracSphRed}
\left( \vec{\alpha}\cdot \vec{p}+ \beta\,m_r + V(r_\mu) \right) \ket{n \kappa m} = (m_r-E_{n \kappa}) \ket{n \kappa m}.
\end{equation}
In relativistic quantum mechanics, this separation is not possible. We follow an approach used in Refs.~\cite{friar1973,BorieRinker1982}, which includes the nonrelativistic part of the recoil correction already in the wave functions by using the reduced mass in the Dirac equation and calculating the leading relativistic corrections perturbatively. If $E^{\text{(fm)}}_{n\kappa}$ denotes the binding energy of Eq.\eqref{eq:diracSph} with the finite size potential from Eq.~\eqref{eq:Hmonopole} but with the reduced mass replaced by the full muon rest mass, and $E^{\text{(rm)}}_{n\kappa}$ the binding energy in the same potential but with the reduced mass from Eq.~\eqref{eq:redmass}, then the leading relativistic recoil correction $\Delta E^{\text{(rec,rel)}}_{n\kappa}$ according to Ref.~\cite{BorieRinker1982} reads
\begin{equation}
\label{eq:relrec}
\Delta E^{\text{(rec,rel)}}_{n\kappa} = -\frac{\left(E^{\text{(fm)}}_{n\kappa}\right)^2}{2 M_N}+\frac{1}{2 M_N}\left< h(r) + 2 E^{\text{(fm)}}_{n\kappa} P_1(r)  \right>,
\end{equation}
where $M_N$ is the mass of the nucleus, and the functions $h(r)$ and $P_1(r)$ are defined in Eqs. (109) and (111) of Ref.~\cite{BorieRinker1982}, respectively. In Table~\ref{tab:recoil}, the binding energies obtained from solving the Dirac equation with the muon rest mass and the reduced mass of the muon-nucleus system are compared, and the leading relativistic recoil correction is shown. The uncertainties include errors in the rms radius, the model of the charge distribution and for the relativistic recoil, and a $(m_\mu/M_N)^2$ term due to higher-order corrections in the mass ratio of muon and nucleus, which dominates the uncertainty for lower $Z$.
%begin table
\begin{table}
\caption{\label{tab:recoil}Recoil corrections to the binding energies of the muon. fm (full mass) denotes the finite size binding energy, analogous to the fourth column of Table~\ref{tab:sphDirac}, but with the rest mass of the muon used in the Dirac equation. $\Delta E_{\text{rec,nr}}$ is the non-relativistic recoil correction, which is the difference between the finite size Dirac solutions with reduced mass and full mass, respectively. $\Delta E^{\text{(rec,rel)}}_{n\kappa}$ is the leading relativistic recoil correction from Section~\ref{sec:recoil}.
All energies are in keV.}
%\begin{ruledtabular}
\centering
\begin{tabular}{lllll}
& state & $E^{\text{(fm)}}$ &$\Delta E^{\text{rec,nr}}$&$\Delta E^{\text{(rec,rel)}}_{n\kappa}$\footnotemark[1]\\ \hline \\[-7pt]
 $^{205}$Bi & 1s\nicefrac{1}{2} & \text{10702.(51.)} & \text{-2.80(4)} & \text{0.39(4)} \\
  & 2s\nicefrac{1}{2} & \text{\phantom{1}3656.(15.)} & \text{-1.42(2)} & \text{0.09(3)}\\
  & 2p\nicefrac{1}{2} & \text{\phantom{1}4895.6(3.0)} & \text{-2.24(1)} & \text{0.12(3)} \\
  & 2p\nicefrac{3}{2} & \text{\phantom{1}4708.2(4.6)} & \text{-2.27(1)} & \text{0.01(1)} \\
  & 3s\nicefrac{1}{2} & \text{\phantom{1}1796.6(5.5)} & \text{-0.78(1)} & \text{0.03(3)} \\
  & 3p\nicefrac{1}{2} & \text{\phantom{1}2180.0(0.5)} & \text{-1.05} & \text{0.03(3)} \\
  & 3p\nicefrac{3}{2} & \text{\phantom{1}2131.9(1.3)} & \text{-1.06} & \text{0.03(3)} \\
  & 3d\nicefrac{3}{2} & \text{\phantom{1}2218.1(0.3)} & \text{-1.21} & \text{0.02(2)} \\
  & 3d\nicefrac{5}{2} & \text{\phantom{1}2174.0(0.2)} & \text{-1.19} & \text{0.02(2)} \\[7pt]
 $^{147}$Sm & 1s\nicefrac{1}{2} & \text{\phantom{1}7168.(28.)} & \text{-3.17(4)} & \text{0.29(7)} \\
  & 2s\nicefrac{1}{2} & \text{\phantom{1}2231.1(6.7)} & \text{-1.31(1)} & \text{0.05(5)} \\
  & 2p\nicefrac{1}{2} & \text{\phantom{1}2779.4(1.5)} & \text{-1.97(1)} & \text{0.05(5)} \\
  & 2p\nicefrac{3}{2} & \text{\phantom{1}2691.2(1.8)} & \text{-1.96(1)} & \text{0.04(4)} \\
  & 3s\nicefrac{1}{2} & \text{\phantom{1}1062.0(2.3)} & \text{-0.68(1)} & \text{0.02(2)} \\
  & 3p\nicefrac{1}{2} & \text{\phantom{1}1229.5(0.4)} & \text{-0.89} & \text{0.01(1)} \\
  & 3p\nicefrac{3}{2} & \text{\phantom{1}1205.6(0.6)} & \text{-0.89} & \text{0.01(1)} \\
  & 3d\nicefrac{3}{2} & \text{\phantom{1}1222.3(0.1)} & \text{-0.93} & \text{0.01(1)} \\
  & 3d\nicefrac{5}{2} & \text{\phantom{1}1208.3} & \text{-0.92} & \text{0.01(1)} \\[7pt]
 $^{89}$Zr & 1s\nicefrac{1}{2} & \text{\phantom{1}3646.5(8.2)} & \text{-3.36(3)} & \text{0.15(15)} \\
  & 2s\nicefrac{1}{2} & \text{\phantom{1}1022.4(1.5)} & \text{-1.11(1)} & \text{0.02(2)} \\
  & 2p\nicefrac{1}{2} & \text{\phantom{1}1149.2(0.2)} & \text{-1.43} & \text{0.01(1)} \\
  & 2p\nicefrac{3}{2} & \text{\phantom{1}1128.4(0.2)} & \text{-1.41} & \text{0.01(1)} \\
  & 3s\nicefrac{1}{2} & \text{\phantom{11}470.3(0.5)} & \text{-0.54} & \text{0.01(1)} \\
  & 3p\nicefrac{1}{2} & \text{\phantom{11}508.6(0.1)} & \text{-0.64} & \text{0.00} \\
  & 3p\nicefrac{3}{2} & \text{\phantom{11}502.7(0.1)} & \text{-0.63} & \text{0.00} \\
  & 3d\nicefrac{3}{2} & \text{\phantom{11}503.7} & \text{-0.64} & \text{0.00} \\
  & 3d\nicefrac{5}{2} & \text{\phantom{11}501.3} & \text{-0.63} & \text{0.00} \\
\end{tabular}
%\end{ruledtabular}
\footnotetext[1]{$\Delta E^{\text{rec,nr}}:=E^{\text{(red.mass)}}-E^{\text{(fm)}}$, see Section~\ref{sec:recoil} for definitions.}
\end{table}
%end table
%
%
\subsubsection{Electron screening}
\label{sec:screen}
%begin table
\begin{table}
\caption{\label{tab:screen}Electron screening corrections to the bound muon energy levels. $\Delta E_{\rm{S,eff}}^{(1)}$ and $\Delta E_{\rm{S,eff}}^{(1+2)}$ are the screening corrections with the effective nuclear charge method, whereas $\Delta E_{\rm{S,3step}}^{(1)}$ and $\Delta E_{\rm{S,3step}}^{(1+2)}$ use the 3 step calculation, both described in Section~\ref{sec:screen}. For the superscript $(1)$, only the 1s electrons are considered, while for $({1}{+}{2})$, all electrons from the first and second shell are considered. All energies are in keV.}
%\begin{ruledtabular}
\centering
\begin{tabular}{ccrrrr}
&$\mu$-state & $\Delta E_{\rm{S,eff}}^{(1)}$  & $\Delta E_{\rm{S,eff}}^{(1+2)}$ & $\Delta E_{\rm{S,3step}}^{(1)}$ & $\Delta E_{\rm{S,3step}}^{(1+2)}$\\ \hline \\[-7pt]
 $^{205}$Bi & 1s\nicefrac{1}{2} & \text{5.555} & \text{10.825} & \text{5.555} & \text{10.825} \\
  & 2s\nicefrac{1}{2} & \text{5.537} & \text{10.803} & \text{5.538} & \text{10.805} \\
  & 2p\nicefrac{1}{2} & \text{5.548} & \text{10.817} & \text{5.549} & \text{10.818} \\
  & 2p\nicefrac{3}{2} & \text{5.547} & \text{10.816} & \text{5.548} & \text{10.817} \\
  & 3s\nicefrac{1}{2} & \text{5.490} & \text{10.748} & \text{5.494} & \text{10.753} \\
  & 3p\nicefrac{1}{2} & \text{5.514} & \text{10.776} & \text{5.516} & \text{10.779} \\
  & 3p\nicefrac{3}{2} & \text{5.512} & \text{10.774} & \text{5.515} & \text{10.777} \\
  & 3d\nicefrac{3}{2} & \text{5.526} & \text{10.791} & \text{5.528} & \text{10.793} \\
  & 3d\nicefrac{5}{2} & \text{5.525} & \text{10.789} & \text{5.527} & \text{10.792} \\[7pt]
 $^{147}$Sm & 1s\nicefrac{1}{2} & \text{3.705} & \text{7.312} & \text{3.705} & \text{7.312} \\
  & 2s\nicefrac{1}{2} & \text{3.699} & \text{7.305} & \text{3.700} & \text{7.305} \\
  & 2p\nicefrac{1}{2} & \text{3.703} & \text{7.309} & \text{3.703} & \text{7.309} \\
  & 2p\nicefrac{3}{2} & \text{3.703} & \text{7.309} & \text{3.703} & \text{7.309} \\
  & 3s\nicefrac{1}{2} & \text{3.682} & \text{7.285} & \text{3.683} & \text{7.286} \\
  & 3p\nicefrac{1}{2} & \text{3.689} & \text{7.293} & \text{3.691} & \text{7.295} \\
  & 3p\nicefrac{3}{2} & \text{3.689} & \text{7.293} & \text{3.690} & \text{7.294} \\
  & 3d\nicefrac{3}{2} & \text{3.694} & \text{7.299} & \text{3.695} & \text{7.300} \\
  & 3d\nicefrac{5}{2} & \text{3.694} & \text{7.298} & \text{3.694} & \text{7.299} \\[7pt]
 $^{89}$Zr & 1s\nicefrac{1}{2} & \text{2.214} & \text{4.405} & \text{2.214} & \text{4.405} \\
  & 2s\nicefrac{1}{2} & \text{2.212} & \text{4.402} & \text{2.212} & \text{4.403} \\
  & 2p\nicefrac{1}{2} & \text{2.213} & \text{4.403} & \text{2.213} & \text{4.403} \\
  & 2p\nicefrac{3}{2} & \text{2.213} & \text{4.403} & \text{2.213} & \text{4.403} \\
  & 3s\nicefrac{1}{2} & \text{2.205} & \text{4.395} & \text{2.206} & \text{4.396} \\
  & 3p\nicefrac{1}{2} & \text{2.207} & \text{4.397} & \text{2.208} & \text{4.398} \\
  & 3p\nicefrac{3}{2} & \text{2.207} & \text{4.397} & \text{2.208} & \text{4.398} \\
  & 3d\nicefrac{3}{2} & \text{2.209} & \text{4.399} & \text{2.210} & \text{4.400} \\
  & 3d\nicefrac{5}{2} & \text{2.209} & \text{4.399} & \text{2.209} & \text{4.400} \\

\end{tabular}
%\end{ruledtabular}
\end{table}
%end table
The effect of the surrounding electrons on the binding energies of the muon was estimated following Ref.~\cite{vogel1973} by calculating an effective screening potential from the charge distribution of the electrons as
\begin{equation}
\label{eq:screenPot}
V_{e}(\vec{r}_\mu)=-\alpha \int \mathrm{d}V\frac{\rho_e (\vec{r})}{|\vec{r}_\mu-\vec{r}|},
\end{equation}
and using this potential in the Dirac equation for the muon. The charge distribution of the electrons is obtained by their Dirac wave functions as $\rho_e (\vec{r})=\sum_i \psi_{e_i}^*(\vec{r})\cdot \psi_{e_i}(\vec{r})$, where $\psi_{e_i}(\vec{r})$  is the four component spinor of the $i$-th considered electron. In order to obtain the wave functions of the electrons, it has to be taken into account, that the muon essentially screens one unit of charge from the nucleus. The simplest possibility is to replace the nuclear charge by an effective charge $\tilde{Z}=Z-1$ and then solve the Dirac equation for the electron with this modified nuclear potential. Another possibility is to start solving the Dirac equation for the muon in the nuclear potential without electron screening. Then, the Dirac equation for the electron is solved for all required states, adding the screening potential due to the bound muon
\begin{equation}
V_{\mu}(\vec{r}_e)=-\alpha \int \mathrm{d}V\frac{\psi_\mu^*(\vec{r})\cdot \psi_\mu(\vec{r})}{|\vec{r}_e-\vec{r}|},
\end{equation}
analogously to Eq.~\eqref{eq:screenPot}.
The interaction between the electrons is not taken into account here. Finally, the Dirac equation for the muon is solved again, now including the nuclear potential and the screening potential \eqref{eq:screenPot} due the atomic electrons from the considered electron configuration. This procedure can be repeated in the spirit of Hartree's method~\cite{bethe_salpeter} until the electrons and the muon are self-consistent in the fields of each other, but our studies show that one iteration is usually enough since the overlap of muon and electron wave functions in heavy muonic atoms is small. It is important to note, that here the screening potential depends to a small extent on the state of the muon, since the muon wave function is used in the calculation for the electron wave function. The atomic electrons primarily behave like a charged shell around the muon and the nucleus; thus every muon level is mainly shifted by a constant term, which is not observable in muonic transitions. The screening correction $\Delta E_S$ is defined as the difference of the binding energy without screening potential and with screening potential, therefore a positive value indicates that the muon is less bound due to the screening effect. The main contribution to the nonconstant part of the screening potential comes from the 1$s$ electrons, since their wave functions have the biggest overlap with the muon; therefore the exact electron configuration has only a minor effect on transition energies~\cite{vogel1973}. In Table~\ref{tab:screen}, results for the screening correction are shown for both mentioned methods and for different electron configurations. Values of the screening correction for different electron configurations show that a 10\% error for the non-constant part is a reasonable estimate.
%
%
%\section{Hyperfine interactions}
%\label{sec:hfs}
\subsubsection{Electric quadrupole splitting}
\label{sec:elQuad1}
%begin hfs table
\begin{table*}
\begin{small}
\caption{\label{tab:hfs}
Results for the electric quadrupole and magnetic dipole hyperfine splitting for a selection of hyperfine states of muonic $^{205}_{83}$Bi ($I=\frac{9}{2}$), $^{147}_{62}$Sm ($I=\frac{7}{2}$), and $^{89}_{40}$Zr ($I=\frac{9}{2}$). $\braket{H_E^{(2)}}$ are the values of the electric quadrupole splitting. $\braket{H_M^{\rm{hom}}}$ is the magnetic dipole splitting from Eq.~\ref{eq:hmag} using a homogeneous nuclear current distribution and $\braket{H_M^{\rm{sp}}}$ using the nuclear magnetization distribution in the single particle model. See Sections~\ref{sec:elQuad1} and~\ref{sec:magndip} for definitions. All energies are in keV.}
%\begin{ruledtabular}
\centering
\begin{tabular}{ccllllll}
 nucleus&state&\multicolumn{2}{c}{$\braket{H_E^{(2)}}$}&\multicolumn{2}{c}{$\braket{H_M^{\rm{hom}}}$}&\multicolumn{2}{c}{$\braket{H_M^{\rm{sp}}}$}\\
 & &$F=I-\frac{1}{2}$&$F=I+\frac{1}{2}$&$F=I-\frac{1}{2}$&$F=I+\frac{1}{2}$&$F=I-\frac{1}{2}$&$F=I+\frac{1}{2}$\\[2pt] \hline \\[-7pt]
   $^{205}$Bi & 1s\nicefrac{1}{2} & \phantom{-11}0 & \phantom{-11}0 & -2.27(20) &\phantom{-}1.86(16) & -2.41(20) &\phantom{-}1.97(16) \\
  & 2s\nicefrac{1}{2} & \phantom{-11}0 & \phantom{-11}0 & \text{-0.43(5)} &\phantom{-}0.35(4) & -0.47(6) &\phantom{-}0.38(4) \\
  & 2p\nicefrac{1}{2} & \phantom{-11}0 & \phantom{-11}0 & -1.23(11) & \phantom{-}1.01(9) & -1.31(11) &\phantom{-}1.07(10) \\
  & 2p\nicefrac{3}{2} & -175.(24.) & \phantom{-}175.(24.) & -0.55(2) & \phantom{-}0.010(4) & -0.554(22) & \phantom{-}0.098(4) \\
  & 3s\nicefrac{1}{2} & \phantom{-11}0 & \phantom{-11}0 & \text{-0.144(20)} & \phantom{-}0.118(16) & -0.160(20) & \phantom{-}0.131(16) \\
  & 3p\nicefrac{1}{2} & \phantom{-11}0 & \phantom{-11}0 & -0.311(33) & \phantom{-}0.255(26) & -0.336(33) & \phantom{-}0.275(27) \\
  & 3p\nicefrac{3}{2} & \phantom{1}-48.9(8.0) & \phantom{-1}48.9(8.0) & -0.160(7) & \phantom{-}0.028(1) & -0.163(7) & \phantom{-}0.029(1) \\
  & 3d\nicefrac{3}{2} & \phantom{1}-25.4(1.3) & \phantom{-1}25.4(1.3) & -0.161(6) & \phantom{-}0.028(1) & -0.163(6) & \phantom{-}0.029(1) \\
  & 3d\nicefrac{5}{2} & \phantom{-1}28.3(1.3) & \phantom{1}-28.3(1.3) & -0.103(3) & -0.027 & -0.103(3) & -0.027 \\[7pt]
  $^{147}$Sm & 1s\nicefrac{1}{2} & \phantom{-11}0 & \phantom{-11}0 & \text{\phantom{-}0.42(18)} & -0.33(14) & \phantom{-}0.25(17) & -0.20(14) \\
  & 2s\nicefrac{1}{2} & \phantom{-11}0 & \phantom{-11}0 & \phantom{-}0.072(39) & -0.056(30) & \phantom{-}0.033(39) & -0.026(30) \\
  & 2p\nicefrac{1}{2} & \phantom{-11}0 & \phantom{-11}0 & \phantom{-}0.164(58) & -0.127(45) & \phantom{-}0.106(58) & -0.082(45) \\
  & 2p\nicefrac{3}{2} & \phantom{1}-32.8(3.2) & \phantom{-1}32.8(3.2) & \phantom{-}0.066(8) & -0.004(1) & \phantom{-}0.058(8) & -0.004(1) \\
  & 3s\nicefrac{1}{2} & \phantom{-11}0 & \phantom{-11}0 & \phantom{-}0.023(13) & -0.018(10) & \phantom{-}0.010(13) & -0.008(8) \\
  & 3p\nicefrac{1}{2} & \phantom{-11}0 & \phantom{-11}0 & \phantom{-}0.044(18) & -0.034(14) & \phantom{-}0.026(18) & -0.02(1) \\
  & 3p\nicefrac{3}{2} & \phantom{11}-9.4(1.1) & \phantom{-11}9.4(1.1) & \phantom{-}0.020(3) & -0.001 & \phantom{-}0.017(3) & -0.001 \\
  & 3d\nicefrac{3}{2} & \phantom{11}-3.2(0.1) & \phantom{-11}3.2(0.1) & \phantom{-}0.015(1) & \phantom{-}0.000 & \phantom{-}0.014(1) & \phantom{-}0.000 \\
  & 3d\nicefrac{5}{2} & \phantom{-11}3.7(0.2) & \phantom{11}-3.7(0.2) & \phantom{-}0.010 & \phantom{-}0.004 & \phantom{-}0.010 & \phantom{-}0.004 \\[7pt]
 $^{89}$Zr & 1s\nicefrac{1}{2} & \phantom{-11}0 & \phantom{-11}0 & \phantom{-}0.36(13) & -0.29(10) & \phantom{-}0.23(12) & -0.19(10) \\
  & 2s\nicefrac{1}{2} & \phantom{-11}0 & \phantom{-11}0 & \phantom{-}0.053(23) & -0.043(18) & \phantom{-}0.030(23) & -0.025(18) \\
  & 2p\nicefrac{1}{2} & \phantom{-11}0 & \phantom{-11}0 & \phantom{-}0.071(14) & -0.058(11) & \phantom{-}0.057(14) & -0.047(11) \\
  & 2p\nicefrac{3}{2} & \phantom{-1}12.2(4.7) &\phantom{1}-12.2(4.7) & \phantom{-}0.023(1) & -0.004 & \phantom{-}0.022(1) &-0.004 \\
  & 3s\nicefrac{1}{2} & \phantom{-11}0 & \phantom{-11}0 & \phantom{-}0.016(7) & -0.013(6) & \phantom{-}0.009(7) & -0.007(6) \\
  & 3p\nicefrac{1}{2} & \phantom{-11}0 & \phantom{-11}0 & \phantom{-}0.020(4) & -0.017(4) & \phantom{-}0.016(4) & -0.013(4) \\
  & 3p\nicefrac{3}{2} & \phantom{-11}3.6(1.4) & \phantom{11}-3.6(1.4) & \phantom{-}0.007 & -0.001 & \phantom{-}0.007 & -0.001 \\
  & 3d\nicefrac{3}{2} & \text{\phantom{-11}0.9(0.3)} & \text{\phantom{11}-0.9(0.3)} & \phantom{-}0.004 & \phantom{-}0.000 & \phantom{-}0.004 & \phantom{-}0.000 \\
  & 3d\nicefrac{5}{2} & \text{\phantom{11}-1.1(0.4)} & \text{\phantom{-11}1.1(0.4)} & \phantom{-}0.003 & \phantom{-}0.000 & \phantom{-}0.003 &\phantom{-}0.000 \\

\end{tabular}
%\end{ruledtabular}
\end{small}
\end{table*}
% end hfs table
Since for heavy nuclei the nuclear radius is comparable to the muon's Compton wavelength~\cite{Angeli2013,codata}, the muonic wavefunction overlaps strongly with the nucleus and the muon is sensitive to nuclear shape corrections, which results in hyperfine splitting of the energy levels. The quadrupole part of the electric interaction \eqref{eq:quadInt} can be rewritten as~\cite{kozhedub2008}
\begin{equation}
\label{eq:Hquad}
H^{(2)}_E = - \alpha \frac{Q_0 F_{\text{QD}}(r_\mu)}{2\, r_\mu^3} \sum_{m=-2}^2 C_{2m}(\vartheta_N,\varphi_N)C_{2m}^{*}(\vartheta_\mu,\varphi_\mu),
\end{equation}
where $C_{lm}(\vartheta,\varphi)=\sqrt{4\pi/(2l+1)}Y_{lm}(\vartheta,\varphi)$ and angles with a subscript $\mu$($N$) describe the position of the muon ($z$ axis of the nuclear frame) in the laboratory frame. Here, the nuclear intrinsic quadrupole moment is defined via the charge distribution from Eq.~\eqref{eq:rho} as
\begin{equation}
\label{eq:defQ0}
Q_0 = 2 \sqrt{\frac{4\pi}{5}} \int_0^\infty r^4 \rho_2(r)\,\mathrm{d}r,
\end{equation}
and the distribution of the quadrupole moment is described by the function $f(r_\mu)$, where in the point-like limit $f(r_\mu)=1/r_\mu^{-3}$. For the shell model, where the quadrupole distribution is concentrated around the nuclear rms radius $R_N$, the divergence for $r_\mu=0$ is removed, and the corresponding quadrupole distribution function is
\begin{equation}
F_{\text{QD}}(r_\mu)=
\begin{cases}
\left(\nicefrac{r_\mu}{R_N}\right)^5 & r_\mu \leq R_N\\
1 &r_\mu > R_N
\end{cases}.
\end{equation}
Formally, this corresponds to a charge distribution with
\begin{equation}
\rho_2(r_\mu)=\frac{Q_0}{2 R_N^4}\sqrt{\frac{5}{4\pi}}\delta(r_\mu-R_N).
\end{equation}
The matrix elements of the quadrupole interaction from Eq.~\eqref{eq:Hquad} in the states~\eqref{eq:totalState} read~\cite{Korzinin2005}
\begin{IEEEeqnarray}{l}
\label{eq:hquad}
\bra{FM_FI\kappa}H^{(2)}_E\ket{FM_FI\kappa}= - \alpha (-1)^{j+I+F}\\
\qquad\quad\times \bra{I|}\frac{Q_0}{2} \widehat{C}_2(\vartheta_N,\varphi_N)\ket{|I}
\bra{n\kappa|}\frac{F_{\text{QD}}(r_\mu)}{r_\mu^{3}}\widehat{C}_2(\vartheta_\mu,\varphi_\mu)\ket{|n\kappa}\nonumber.
\end{IEEEeqnarray}
The reduced matrix element in the nuclear coordinates can be expressed with the spectroscopic nuclear quadrupole moment $Q$ as
\begin{equation}
\nonumber
\bra{I|}\frac{Q_0}{2} \widehat{C}_2(\vartheta_N,\varphi_N)\ket{|I}= Q\sqrt{\frac{(2I+3)(2I+1)(I+1)}{4I(2I-1)}},
\end{equation}
and the reduced matrix elements in the muonic coordinates are
\begin{IEEEeqnarray}{l}
\bra{n\kappa|}f(r_\mu)\widehat{C}_2(\vartheta_\mu,\varphi_\mu)\ket{|n\kappa} =\\[5pt]
\quad-\sqrt{\frac{(2j+3)(2j+1)(2j-1)}{16j(j+1)}}
\int_0^\infty \left( G^2_{n\kappa}(r_\mu)+F^2_{n\kappa}(r_\mu)\right)\frac{F_{\text{QD}}(r_\mu)}{r_\mu^{3}} \mathrm{d}r_\mu.\nonumber
\end{IEEEeqnarray}
The values for the nuclear rms-radii $R_N$ and the spectroscopic quadrupole moments $Q$ are taken from Refs.~\cite{Angeli2013,Stone2005}. In Table~\ref{tab:hfs}, results for the electric quadrupole hyperfine splitting for the nuclei $^{205}_{83}$Bi, $^{147}_{62}$Sm, and $^{89}_{40}$Zr are shown for a selection of hyperfine states, including uncertainties stemming from the error in the quadrupole moment and an estimation of the modeling uncertainty.
%
%
\subsubsection{Magnetic dipole splitting}
\label{sec:magndip}
As for the magnetic part, we consider dipole interaction. Therefore, the corresponding interaction Hamiltonian from Eq.~\eqref{eq:Hint} reads~\cite{Elizarov2005}
\begin{equation}
\label{eq:Hmag}
H_{M} = \frac{|e|}{4 \pi}\,\vec{\mu}\cdot \left( F_{\text{BW}}(r) \frac{\vec{r}}{r^3} \times \vec{\alpha} \right),
\end{equation}
with the charge of the muon $e=-|e|$, the nuclear magnetic moment $\vec{\mu}$, its distribution function $F_{\text{BW}}$, and the Dirac matrices $\vec{\alpha}$. If the nuclear current density is described by a normalized scalar function $f_\mu(r)$ as
\begin{equation}
\label{eq:currentdistr}
\vec{j}(r)= \text{rot}\left(\vec{\mu}f_\mu(r)\right),
\end{equation}
then the distribution function is given by
\begin{equation}
\label{eq:Fbw}
F_{\text{BW}}(r)=-r^2 \frac{\partial}{\partial r}\,\int \text{d}V^{\prime}\,\frac{f_\mu(r^{\prime})}{|\vec{r}-\vec{r}\,^{\prime}|}.
\end{equation}
The difference in the hyperfine splitting between a point-like magnetic moment and a spacial distribution of the magnetization is called the Bohr-Weisskopf effect~\cite{bohrWeisskopf1950}. In Sec.~\ref{sec:elQuad1}, the matrix elements of the magnetic interaction are analyzed, paying special attention to the distribution function $F_{\text{BW}}$. We expect the contribution of the higher-order terms, namely electric octupole, magnetic quadrupole, and beyond, to be smaller than the uncertainty of the considered terms~\cite{Devons1995,Steffen1985}. Therefore they can be ignored here.

In addition, the hyperfine splitting arises from the interaction of the nuclear magnetic moment with the muon's magnetic moment, which is also sensitive to the spatial distribution of the nuclear currents. Since the magnetic moment of the muon is inversely proportional to its mass, the magnetic hyperfine splitting in muonic atoms is less important than in electronic atoms. The matrix elements of the corresponding Hamiltonian~\eqref{eq:Hmag} in the state~\eqref{eq:totalState} are~\cite{Korzinin2005}
\begin{IEEEeqnarray}{l}
\label{eq:hmag}
\bra{FM_FI\kappa}H_M\ket{FM_FI\kappa}=
\,\,\left[ F(F+1)-I(I+1)-j(j+1)\right] \\[7.5pt]
\qquad\qquad\qquad\qquad\,\,\times\frac{\alpha}{2 m_p}\frac{\mu}{\mu_N}\frac{\kappa}{Ij(j+1)}\int_0^\infty \frac{G_{n\kappa}(r_\mu)F_{n\kappa}(r_\mu)F_{\text{BW}}(r_\mu)}{r_\mu^2}\mathrm{d}r_\mu,\nonumber
\end{IEEEeqnarray}
where $m_p$ is the proton mass, and the ratio of the observed magnetic moment $\mu:=\bra{II}(\vec{\mu})_z\ket{II}$ and the nuclear magneton $\mu_N$ can be found in the literature~\cite{Stone2005}. For the simple model of a homogeneous nuclear current distribution the distribution function from Eq.~\eqref{eq:Fbw} of the Bohr-Weisskopf effect reads
\begin{equation}
\label{eq:bwsimple}
F_{\rm{BW}}(r_\mu)=\begin{cases}
\left( \nicefrac{r_\mu}{R_N} \right)^3 & r_\mu \leq R_N\\
1 &r_\mu > R_N
\end{cases}.
\end{equation}
Furthermore, an additional method is used to obtain the distribution function $F_{\rm{BW}}$ from the nuclear single particle model, where the nuclear magnetic moment is assigned to the odd nucleon and the Schrödinger equation for this nucleon is solved in the Woods-Saxon potential of the other nucleons~\cite{Elizarov2005}. In Table~\ref{tab:hfs}, results for the magnetic dipole hyperfine splitting for the nuclei $^{205}_{83}$Bi, $^{147}_{62}$Sm, and $^{89}_{40}$Zr are presented for a selection of hyperfine states, using both methods for obtaining $F_{\rm{BW}}$, where the model error is estimated by the difference of these two methods and the uncertainty in the magnetic moment is also taken into account.
%

\subsection{Dynamic hyperfine structure}
\label{sec:muon_dynamic}

\subsection{Residual second order corrections}
\label{sec:muon_residualSO}

\subsection{Quadrupole-Uehling interactions}
\label{sec:muon_quadUehl}

\section{Struture of muonic Rhenium-185}
\label{sec:muon_re}

\subsection{Low lying states}
\subsection{Extraction of quadrupole moment from $5g\rightarrow 4f$ transitions}



\section{Bound muon g factor in Helium-4}
\label{sec:muon_he}

\section{Summary}
\label{sec:muon_summary}
