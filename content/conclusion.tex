\chapter*{Summary \& Outlook}
\markboth{Summary \& Outlook}{}
\label{ch:conclusion}


In the present thesis, nuclear structure effects in the spectra of heavy ions and muonic atoms caused by extended and deformed nuclear charge distributions are investigated. 
Here, the focus is on two topics, namely on the one hand improved calculations of the nuclear shape effect on the bound-electron g factor for spinless nuclei beyond the previously used perturbative evaluation, and on the other hand the analysis of the level structure of muonic atoms.\\

Chapter~\ref{ch:nucl_def} is about non-perturbative calculations of nuclear shape effects on the bound-electron g factor. Here, the previously used perturbative method is introduced, which is called the effective radius method because the radius of a homogeneously charged sphere with approximately the same energy correction as the deformed nuclear charge dustribution is used. 
Then, the non-perturbative, numerical method used in this thesis is explained, wherein the nuclear potential, the solution of the Dirac equation and the corresponding g factor are calculated all-numeric, starting with the deformed nuclear charge distribution. By performing calculations for a wide range of nuclei across the nuclear chart, it is shown that the perturbative evaluation overestimated the nuclear shape effect on the 20\% level. The difference of the numerical and perturbative, effective radius method is investigated. The formulas for perturbative calculation of effective radius and corresponding energy correction of the homogeneously charged sphere are mainly responsible for the disagreement, but also the incompleteness of the effective radius method itself is visible.

Furthermore, it is demonstrated how the consideration of deformed nuclei can reduce the uncertainty in the theoretical prediction of finite-nuclear-size effects on the bound-electron g factor. The previous, conservative estimation of this uncertainty is the difference in the g factors due to a homogeneously charged sphere and a Fermi-type nuclear charge distribution. If parameters of the deformed nuclear charge distribution are available, the finite nuclear size and shape g-factor corrections can be calculated from thereon. In this way, the remaining model uncertainty is reduced to the uncertainty due to the parameters describing the deformed nuclear charge distribution. For hydrogenlike Uranium, it is shown that in this way the uncertainty of finite nuclear size effects can be halved.\\ 

In Chapter~\ref{ch:muonic_atoms}, the spectra of muonic atoms are considered.\\

ideas outlook:
\begin{itemize}
\item crosscheck charge distribution parameters from muonic experiments and use these in g factor experiments to check consistency of electronic and muonic systems
\item team up with nuclear physics for new nuclear polarization corrections and thereby analysis of low-lying states
\end{itemize}
 






