\chapter*{Summary \& Outlook}
\markboth{Summary \& Outlook}{}
\label{ch:conclusion}

%\subsubsection*{Summary}
In the present thesis, nuclear structure effects caused by extended and deformed nuclear charge distributions and corrections from quantum electrodynamics in the spectra of heavy ions and muonic atoms are investigated. 
Here, the focus is on two topics, namely, on the analysis of the level structure and spectra of muonic atoms, and on improved calculations of the nuclear shape effect on the bound-electron $g$ factor for spinless nuclei beyond the previously used perturbative evaluation.\\[11pt]%
%
Chapter~\ref{ch:muonic_atoms} deals with high-precision calculations of the spectra of muonic atoms. 
As a first step, the implementation of the most important effects, namely, finite nuclear size, vacuum polarization, recoil, and electron screening on the fine and hyperfine structure is discussed in Sec.~\ref{sec:calculationSpectraMuon}.
This includes calculations of the dynamical hyperfine structure, which means that the hyperfine structure is considered beyond the first order in the quadrupole interaction for the most important states.
A finite-basis-set method based on B-splines has been used, which is a well established and efficient method in atomic physics, but had not been used in the context of muonic atoms before. Thereby, a practical, numerical representation of the complete spectrum of muon wave functions is obtained.

In Sec.~\ref{sec:higherorder}, enhanced theoretical approaches for calculations connected to the electric quadrupole interaction between muon and nucleus are presented. 
Firstly, this includes a numerical evaluation of the leading-order vacuum polarization correction (Uehling potential) to the quadrupole matrix elements for an arbitrary, deformed nuclear charge distribution. In contrast to previous works, this is done without any approximations on the shape of the charge distribution or the distance between nucleus and muon. For this, a multipole expansion of the Uehling potential is performed. In this thesis, the corresponding expansion coefficients are given in a form suitable for numerical evaluation, as well as analytically in terms of special functions.
Secondly, the energy correction due to residual second-order quadrupole interaction is calculated by means of the finite basis set of muon wave functions.
Both contributions are shown to be potentially important for upcoming experiments.

In Sec.~\ref{sec:muon_re}, the theoretical calculations of this thesis are compared to state-of-the-art experiments in muonic atom spectroscopy, performed recently at the Paul Scherrer Institute (Switzerland) by the MuX collaboration. Theoretical spectra have been fitted to experimental ones by adjusting the parameters of the nuclear model. In this way, the nuclear quadrupole moment of $_{\phantom{1}75}^{185}$Re and $_{\phantom{1}75}^{187}$Re were extracted from the observed ${n}{=}{5}\rightarrow {n}{=}{4}$ x-rays. Obtaining values for the nuclear quadrupole moments is of great importance, because they can be tested against predictions from theoretical nuclear models. 
Also, the spectra of low-lying muonic x-rays in $_{\phantom{1}75}^{185}$Re have been explained by the calculations in this thesis.\\[11pt]%
%
Chapter~\ref{ch:nucl_def} covers non-perturbative calculations of nuclear shape effects on the bound-electron $g$ factor. Here, the previously used perturbative method is introduced, which is called the effective-radius method, because the homogeneously charged sphere model of a given radius with approximately the same energy correction as the deformed nuclear charge distribution is used. 
Then, the non-perturbative, numerical method used in this thesis is explained, wherein the nuclear potential, the solution of the Dirac equation and the corresponding $g$ factor are calculated in an all-numeric manner, starting with the deformed nuclear charge distribution. Calculations for a wide range of nuclei across the nuclear chart reveal that the perturbative evaluation overestimated the nuclear shape effect on the 20\% level. The difference between the fully numerical and perturbative, effective-radius method is investigated. It is shown that the formulas for a perturbative calculation of the effective radius and the corresponding energy correction of the homogeneously charged sphere are mainly responsible for the disagreement, but also the incompleteness of the effective radius method itself contributes.

Furthermore, the connection between nuclear finite size and deformation effects is discussed and it is demonstrated how the consideration of deformed nuclei can reduce the model uncertainty in the theoretical prediction of finite-nuclear-size effects on the bound-electron $g$ factor. The previous, conservative estimation of this uncertainty is the difference in the $g$ factors due to a homogeneously charged sphere and a Fermi-type nuclear charge distribution. If parameters of the deformed nuclear charge distribution are available, the finite-nuclear-size and shape $g$-factor corrections can be calculated with their help. In this way, the remaining model uncertainty is reduced due to the more realistic nuclear model, which is demonstrated with calculations for hydrogen-like uranium.\\[11pt]%
%
Finally, in Chapter~\ref{sec:muon_he}, finite nuclear size and several vacuum polarization corrections to the bound-muon $g$ factor in muonic helium are presented. In combination with other calculations, this enabled a theoretical prediction of the $g$ factor on the $10^{-9}$ level. It has been shown that not only the finite nuclear size and first-order Uehling correction are important on this level of accuracy, but also the two-loop Uehling and Källén-Sabry corrections, evaluated in this thesis. 
It was proposed that an independent and more accurate determination of the muon mass is possible through a combination of our results with future measurements of a similiar accuracy.\\[11pt]%
%
%\subsubsection*{Outlook}
The ongoing experimental campaign on spectroscopy of heavy muonic atoms by the MuX collaboration will provide further possibilities to extract information on atomic nuclei from muonic x-rays, where the methods and codes from this thesis can be used. 
Due to progress in the experiments, it will be possible to analyze muonic x-rays for the first time also for radioactive nuclei, which will include the measurement of muonic x-rays up to the heaviest elements like $_{\phantom{1}96}^{248}$Cm. 
The analysis of low-lying transitions is particularly interesting, since they contain the most information on the nuclear structure, such as the RMS radius of the electric charge distribution.

Currently, the limiting factor for theoretical predictions of low-lying transitions is the nuclear polarization correction. This is a second-order correction to the-bound state energies due to virtual excitation of the atomic nucleus in a muonic atom. Therefore, many excited states of the nucleus contribute and as a consequence, an advanced nuclear model or a complete set of experimental data has to be used for a description of the nuclear polarization correction.
The calculation of the residual second-order quadrupole interaction, as performed in this thesis, already demonstrated how the muonic part in second-order corrections can be evaluated with finite-basis-set methods. It would be highly desirable to combine the muonic calculations of this thesis with up-to-date nuclear structure theory or experimental data for a precise evaluation of the nuclear polarization correction in heavy muonic atoms. However, ab initio calculations for heavy nuclei pose a great challenge for nuclear structure theory.\\[11pt]%
%
Thinking further ahead, it would be insightful to cross-check the consistency of nuclear effects in muonic and electronic atoms in the high-$Z$ regime. 
For example, muonic atom spectroscopy with $_{\phantom{1}96}^{248}$Cm can be expected in the near future and the shape of the nuclear charge distribution can be potentially extracted from the corresponding muonic x-rays. Although this is a radioactive isotope, it has a halflife of several hundreds of thousands of years, thus also Penning trap experiments on the bound-electron $g$ factor in $_{\phantom{1}96}^{248}$Cm might be feasible. Then, the parameters of the nuclear charge distribution obtained from muonic x-rays can be used to calculate the finite nuclear size and nuclear shape effects for the bound-electron $g$ factor. Provided that all other contributions, such as two-loop QED corrections, are under control, a comparison with the measured $g$ factor can test the consistency of nuclear effects in electronic and muonic atoms. 
Furthermore, a better understanding of nuclear structure effects and a higher accuracy of nuclear parameters can lead to more stringent tests of QED, for example via $g$-factor measurements, and to improved determination of fundamental constants.

%According to the principle of lepton universality, electrons and muons couple to fields in the same way and the only difference is their mass.
%
%.\\[1cm]
%ideas outlook:
%\begin{itemize}
%\item MuX will keep measuring up to the very heaviest elements in the near future and for this, updated calculations as presented in this thesis are needed
%\item team up with nuclear physics for new nuclear polarization corrections and thereby analysis of low-lying states
%\item crosscheck charge distribution parameters from muonic experiments and use these in g factor experiments to check consistency of electronic and muonic systems
%\end{itemize}
 






